% vim:fileencoding=utf-8
% -*- encoding: utf-8 -*-
% -*- encoding: latin1 -*-
% vim:fileencoding=latin1
\chapter{Alkoholreklame \textendash{} Hva er lovlig/ulovlig reklame (i praksis)}
\section{Annonser for skjenkesteder}

Et skjenkested har adgang til å annonsere med navn, beliggenhet, bevillingsrettigheter og åpningstider.

Alkoholprodusenters firmanavn/firmamerke kan ikke forekomme i annonsen.

Navn på eller avbildninger av alkoholholdige drikker kan ikke forekomme i annonsen.

Prisopplysninger må ikke forekomme i annonsen. Som prisopplysninger anses også begreper som «Happy Hour»,«Byens billigste» eller liknende betegnelser som informerer om prisene/prisnivået på alkoholholdige drikker.

\section{Plakater og skilt i tilknytning skjenkested}

Utendørs og innendørs plakater, skilt, lysreklame og vindusreklame kan ikke være merket med alkholprodusenters firmanavn/firmamerke, navn på alkoholprodukter eller avbildninger av alkoholholdige drikker

\section{Prisopplysninger på skjenkesteder}

Som en konsekvens av kundenes krav på å få vite hva de enkelte varene koster, er det tillatt å opplyse om prisene på alkoholholdige drikker. Reklameforbudet begrenser imidlertid på hvilken måte og i hvilken form disse prisene kommer til uttrykk.

Utenfor skjenkestedet kan det gis generell prisinformasjon uten å fremheve spesielle produkter, f. eks. øl fra kr 30\textendash60,\textendash

På skjenkestedet kan priser på alkoholholdig drikk bare gis på lik linje med prisinformasjon for andre produkter.

Kravet om at priser på alkoholholdig drikk bare kan gis på lik linje med annen prisinformasjon innebærer at det ikke er tillatt å benytte iøynefallende eller fremtredende

plakater eller effekter for slike prisopplysninger. Det er heller ikke tillatt å bruke suggererende eller salgsfremmende metoder.

Prisopplysninger på alkoholholdige drikker kan bare komme til uttrykk på en nøytral og ren informativ måte. I motsetning til hva som gjelder for andre varer, er det ikke tillatt å opplyse om tilbudspriser, prisavslag eller reduserte priser på alkoholholdige drikker.

Prislister inne i skjenkelokalet kan inneholde alkoholprodusentens firmanavn, for eksempel «1/2 liter Ringenes pils kr 45,\textendash», «1/2 liter Hansa pils kr 45,\textendash».

\section{Gjenstander på skjenkesteder}

I utgangspunktet vil merking av gjenstander med alkoholprodusenters firmanavn/firmamerke samt navn på/avbildninger av alkoholholdige drikker anses som ulovlig reklame for alkoholholdig drikk.

Av hensyn til den praktiske gjennomføringen av alkoholserveringen, er det imidlertid gjort særlige unntak for vanlig serveringsutstyr. Se avsnittet «Unntak fra hovedregelen».

\section{Plassering av alkoholholdig drikk i skjenkelokalet}

Plassering av alkoholholdig drikk er tillatt så lenge denne kan begrunnes ut i fra hensynet til den praktiske gjennomføringen av serveringen.

Dersom skjenkelokalet er lite eller har begrenset lagerkapasitet, er det en videre adgang til å plassere alkoholholdig drikk i selve skjenkelokalet, enn for skjenkesteder med store lagringsmuligheter.

Hvorvidt plasseringen er lovlig må avgjøres i hvert enkelt tilfelle. Ved vurderingen må hensikten med plasseringen og individuelle forhold på stedet tas i betraktning.

I tilfelle hvor alkoholholdige drikker er plassert på en slik måte at den går over til å være spesiell utstilling, anses dette som ulovlig reklame. Generelt kan spesiell utstilling beskrives som en ekstraordinær fremheving av alkoholprodukter. Selv om plasseringen forklares med at den er til pynt, hygge eller dekorasjon, er den likevel ulovlig.

Reglene for annonser, plakater, skilt, plassering og prisopplysninger på salgssteder er i hovedsak identiske med de regler som gjelder for skjenkesteder.

\section{Prisopplysninger}

Opplysninger om priser på alkoholholdige drikker kan bare gis inne i salgslokalet.

Spesielle regler gjelder for salg via internett. Se avsnittet «Unntak fra lovens hovedregel».

Opplysninger om priser på alkoholholdige drikker kan bare gis på lik linje med prisopplysninger for andre varer.

Prislister inne i salgslokalet kan inneholde alkoholprodusenters firmanavn for eksempel: «0,33 l Ringnes pils kr 15,\textendash»,«0,33 l Hansa pils kr 15,\textendash».

Prislistene skal være utformet og plassert på samme måte som prislister for salgsstedets øvrige varer. Det er ikke tillatt å annonsere eller på annen måte tilkjennegi reduserte priser eller tilbudspriser på alkoholholdige drikker, verken i annonser eller via plakater i eller utenfor salgsstedet.

Prislister for alkoholholdig drikk kan ikke inneholde alkoholprodusenters firmamerke.

\section{Plakater og skilt plassert utenfor salgslokalet}

Skilt/plakater utenfor salgsstedet kan opplyse om at det selges øl på stedet. Opplysningen om at det selges øl skal komme til uttrykk på en nøytral og informativ måte. Det er ikke tillatt å opplyse om hvilke ølmerker som selges på stedet.

Skilt/plakater plassert utenfor salgslokalet kan ikke inneholde prisopplysninger eller alkoholprodusenters firmanavn/firmamerke.

Skilt/plakater utenfor salgssted kan ikke inneholde navn på bestemte alkoholholdige drikker eller avbildninger av slike.

\section{Plakater og skilt plassert inne i salgslokalet}

Skilt/plakater inne i salgslokalet kan inneholde informative opplysninger om hvilke ølmerker (alkoholprodusentens navn) som selges på stedet og prisene på disse uten at noen produkter fremheves i forhold til andre.

Andre typer innendørs plakater, skilt, lysreklame og lignende, kan ikke merkes med alkoholprodusenters firmanavn/firmamerke, navn på alkoholprodukter eller avbildninger av alkoholholdige drikker.

Alkoholreklame kan heller ikke forekomme på kjøleskap eller liknende gjenstander.

\section{Plassering av alkoholholdig drikk i salgslokalet}

Plassering av alkoholholdig drikk er tillatt så lenge denne kan begrunnes ut i fra hensynet til den praktiske gjennomføringen av salgsvirksomheten.

Hvorvidt plasseringen er lovlig må avgjøres i hvert enkelt tilfelle. Ved vurderingen må hensikten med plasseringen og individuelle forhold på stedet tas i betraktning.

I tilfelle hvor alkoholholdige drikker er plassert på en slik måte at den går over til å være spesiell utstilling, anses dette som ulovlig reklame. Generelt kan spesiell utstilling beskrives som en ekstraordinær fremheving av alkoholprodukter. Selv om plasseringen forklares med at den er til pynt, hygge eller dekorasjon, er den likevel ulovlig.

Plassering som tydelig tar sikte på å bli sett utenfra er ikke tillatt. 
