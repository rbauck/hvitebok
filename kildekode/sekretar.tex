% vim:fileencoding=utf-8
% -*- encoding: utf-8 -*-
\section{Stillingsbeskrivelse for sekretær}

Sekretær er nestleder og Formands stedfortreder.
Han/hun skal så langt det er mulig assistere og
avlaste Formand i hans/hennes arbeid.

Sekretæren skal i tillegg:
\begin{itemize}
\item Føre referat fra alle Hovedstyremøter og
      sende til Hovedstyret innen påfølgende møte. 
	\subitem Det er også vanlig at sekretær tar ansvar for at det er kaffe på møtene, men det må hvert enkelt sittende styre finne ut av på egenhånd.
	\subitem Formand og sekretær finner vanligvis ut av hvem som skal ha ansvar for å sende ut innkalling i forkant av hver Hovedstyremøte, men en dyktig sekretær vil alltid ha oversikten, og bør uansett minne Formand på god tid i forveiene dersom han/hun skulle glemme det. 
\item Sørge for at det til enhver tid finnes nødvendig kontorutstyr på kontoret. Deriblant:
	\subitem stiftemaskin med skifter
	\subitem lamineringsmaskin med papir
	\subitem gull- og sølvark til hhv. evig intern- og livsvarig medlemskap-kort
	\subitem papir i skriveren
	\subitem skrivesaker
	\subitem hullemaskin, saks, osv. 
\item Produsere internkort i begynnelsen av semesteret og underveis når nye kommer til. Dette er spesielt viktig å få ut så fort som mulig dersom det 
skal være mulig å kreve fremvisning av internbevis for å få internpris.
	\subitem Sørg også for at de som har gjort seg fortjent til gullkort får dette ved utgangen av semesteret. Dette er også en passende tid å produsere sølvkort. Høsten 2010 løste vi det sånn at etterhvert som folk betalte inn korrekt beløp til RF-kontoen, sendte ØU-leder beskjed til sekretær om at innbetalingen var registrert og at sølvkort kunne produseres. 
\item Oppdatere internlister dersom internansvarlig mangler.
\item Det er vanlig skikk at Hovedstyret bemanner baren én gang i semesteret (for å være greie med de internansvarlige). Sekretæren må gjerne minne Hovedstyret på at dette kan gjøres.
\item Det er underforstått at sekretær melder seg frivillig til å skrive referat på generalforsamlingen, og det er også vanlig at sekretær stiller i skjørt (uavhengig av kjønn) ved denne anledningen.
\item Selv om det ikke er enkelte kurs arrangert av SiO som egner seg spesielt godt for sekretær, kan det godt være nyttig å dra på f.eks. profil- og omdømmekurs. (Selv om Formand og sekretær høsten 2010 har litt negative erfaringer med akkurat dette kurset.) Det er aldri dumt å dra på mange forskjellige kurs!
\item Kreve inn, levere ut og ha oversikt over kjellernøkler og nøkkelkort til RF-kontoret. Det kan være lurt å ringe rundt til de du ikke ser til daglig, for hvis enkelte ikke får en personlig henvendelse, har de ikke så lett for å gi fra seg nøkler som de egentlig ikke trenger.
\item Skaffe utvidet tilgang til bygget for de som har behov for det. Navn og kortnummer sendes til vaktalarm@admin.uio.no. Dette er noe de forskjellige lederne ikke alltid husker på, så det er smart å sende ut en mail på f.eks. styret-mailinglista for å få folk til å sende deg informasjonen du trenger.
\item Sende inn Cosa Nostra-liste med navn og tilhørende verv til Nostrakortansvarlig (dette går på rundgang mellom kjellerforeningene). 
	\subitem Dette er vervene som automatisk får Nostrakort (per januar 2011): Formand, sekretær, ØU-leder, ØU-sekretær, arrsjef, fagsjef, regiformann, regiforfører, blæstsjef, internansvarlige, skjenkemestre, dagansvarlige, kafésjef, kjellermester
	\subitem Disse vervene kan godt få, men kun etter at personen har vist vilje til god arbeidsinnsats og man er sikker på at de fortjener det: panikkansvarlig, bedpresansvarlig, popvitansvarlig, vaktmester, og andre ildsjeler
\item Dersom Hovedstyret har bestemt at de skal ha kontortid, bør sekretær sørge for at folk møter opp på deres skift til riktig tid. Ellers blir det bråk!  
\end{itemize}
