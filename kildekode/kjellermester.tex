\section{Kjellermester}

Kjellermester er hovedansvarlig for drift av kjelleren. Herunder ligger kafedrift p� dagtid, kjellerkroer og andre arrangementer i kjelleren. 

\subsection{Kjellermesters oppgaver}
\begin{itemize}
\item Bestilling av varer; �l, rusbrus, vin og sprit
\item S�rge for at det alltid er nok av andre n�dvendigheter,
      eks krystallsoda, s�pe til oppvaskmaskin, kluter, s�pe osv.
\item Hold orden p� lagerbeholdningen (til alt utenom kafedrift)
\item Holde orden p� alt som g�r inn og ut av lager
\item F�re Z-rapporter
\item F�re bilag
\item Betale regninger i tide!
\item Holde m�ter i kjellerstyret, og holde seg oppdatert p�
      hva som skjer i de andre styrene (infostyret, arrstyret)
\item Passe p� at de andre gj�r jobben sin.
\item S�rge for at det er nok vekslepenger til en hver tid.
\item V�re med hovedstyret � arrangere kranf�rekurs.
\item V�re med hovedstyret � arrangere internfester.
\item Oppl�ring av neste kjellermester.
\item Oppl�ring av skjenkemestre.
\end{itemize}

\subsection{Under seg har KM}
\begin{itemize}
\item Kaf\'esjef som har med alt det praktiske rundt kafeen � gj�re
      (KM har med andre ord inget reelt ansvar for kaf\'edriften).
\item Skjenkemestre (fire stykker pr i dag). Disse deler ansvaret
      for gjennomf�ring av kjellerkroer, varemottak og organisering
      av Mandager.
\item Utleiesjef. Tar seg av alt det praktiske rundt utleie av kjelleren.
\end{itemize}

\subsection{Hengende ting}
\begin{itemize}
\item Ferdigstilling av kjelleren.
\item Liming av teppefirkanter (rundt baren og i kroken der kj�leskapet stod).
\item Tette sokkelen rundt baren (med silikon).
\item Overbygg over �lslangen.
\item Henge opp s�pedispenser og papirholder.
\item Lage arbeidsinstruks til de forskjellige kjellervervene.
\item Service p� tappeanlegget.
\item Rydding av anti (i samarbeid med kafesjef og arrsjef under ledelse av formann).
\item Riving av toalettene p� Helligste.
\item Kj�ling av Aller Helligste.
\item Rensing av teppet.
\item F� l�s p� alle kj�leskap.
\end{itemize}
