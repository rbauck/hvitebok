% vim:fileencoding=utf-8
% -*- encoding: utf-8 -*-
\section{Kjellermester}

Kjellermester er hovedansvarlig for drift av kjelleren. Herunder ligger kafedrift på dagtid, kjellerkroer og andre arrangementer i kjelleren. 

\subsection{Kjellermesters oppgaver}
\begin{itemize}
\item Bestilling av varer; øl, rusbrus, vin og sprit
\item Sørge for at det alltid er nok av andre nødvendigheter,
      eks krystallsoda, såpe til oppvaskmaskin, kluter, såpe osv.
\item Hold orden på lagerbeholdningen (til alt utenom kafedrift)
\item Holde orden på alt som går inn og ut av lager
\item Føre bilag
\item Gi ØU bilag og fakturaer raskt
\item Holde møter i kjellerstyret, og holde seg oppdatert på
      hva som skjer i de andre styrene (infostyret, arrstyret)
\item Passe på at de andre gjør jobben sin.
\item Sørge for at det er nok vekslepenger til en hver tid.
\item Være med hovedstyret å arrangere kranførekurs.
\item Være med hovedstyret å arrangere internfester.
\item Opplæring av neste kjellermester.
\item Opplæring av skjenkemestre.
\end{itemize}

\subsection{Under seg har KM}
\begin{itemize}
\item Kaf\'esjef som har med alt det praktiske rundt kafeen å gjøre
      (KM har med andre ord inget reelt ansvar for kaf\'edriften).
\item Skjenkemestre (fem stykker pr i dag). Disse deler ansvaret
      for gjennomføring av kjellerkroer, varemottak og organisering
      av Mandager.
\item Utleiesjef. Tar seg av alt det praktiske rundt utleie av kjelleren.
\end{itemize}

\subsection{Hengende ting}
\begin{itemize}
\item Ferdigstilling av kjelleren.
\item Holde arbeidsinstruks til de forskjellige kjellervervene oppdatert.
\item Service på tappeanlegget.
\item Rydding av anti (i samarbeid med kafesjef og arrsjef under ledelse av formann).
\item Kjøling av Aller Helligste.
\item Rensing av teppet.
\item Få lås på alle kjøleskap.
\end{itemize}
\section{Kjellermester}


