% vim:fileencoding=utf-8
% -*- encoding: utf-8 -*-
%Denne fil vil trenge en
%\usepackage{amsmath,url}
%i preamble.
%
% \chapter{Fagstyret}
Om fagstyrets utforming og mandat:
\begin{quote}

    % Ajour per H10
    {\Large § 11 FAGSTYRET}
    
    \begin{itemize}
        \item[a)] Fagstyret ledes av Fagsjef, og har følgende andre medlemmer:
        Populærvitenskapelig sjef og Bedriftskontakt.
        
        \item[b)] Fagstyret har ansvar for foreningens faglige profil og
        kontakt med næringslivet.
        
        \item[c)] Fagstyret skal fremlegge forslag til foreningens faglige
        program for Hovedstyret.
        
        \item[d)] Fagstyret har ansvaret for å rapportere regnskapsrelevant
        informasjon til Økonomiutvalget og overholde vedtatte budsjetter.
    \end{itemize}

\end{quote}

\section{Fagstyrets oppbygning}
\subsection{Fagsjef}
Fagsjef er sjef for fagstyret, og sitter i tillegg i Hovedstyret.
Ved siden av å lede fagstyret er
fagsjefen som regel også enten Pop.Vit.-ansvarlig eller Bedriftskontakt.
I motsetning til de fleste andre verv er fagsjef en ganske fri stilling; Ting
som gjøres skjer oftest på eget initiativ.

\subsection{Bedriftskontakt}
Bedriftskontakten har ansvar for å pleie kontakten mellom arbeidsgivere og
foreningen, bl.a.~ved å arrangere bedriftspresentasjoner og ekskursjoner.

\subsection{Populærvitenskap-ansvarlig}
Pop.Vit.-ansvarlig har hovedansvar for å skaffe foredragsholdere og debattanter,
og å finne relevante temaer.

\subsection{Panikkansvarlig}
Panikkansvarlig har ansvar for å arrangere panikkhjelp.

\section{Ansvarsområder}
\subsection{Foreningens faglige profil}
\begin{quote}
    % Allerede de gamle grekere?
	Allerede de aller tidligste lover slo fast at foreningens formål er delt: en
	faglig og en sosial del. De tidligste møter i foreningen ble innledet
	med foredrag og debatter før de gikk over i fest og moro. Selv om den
	faglige delen i dag står noe svakere, er den en viktig del av
	foreningen.
\end{quote}

For foreningens faglige profil arrangeres det fortsatt foredrag, kurs og andre
ting.

For å arrangere noe kreves følgende:
\begin{itemize}
	\item Personer, for eksempel foredragsholdere. Dette er den mest
	krevende faktoren.
	\item Lokale. 
          Kjelleren er et fint lokale, den inneholder alt du ønsker av
          mikrofoner, prosjektorer etc. Ta kontakt med utlånsansvarlig i KS.
          Ellers kan Universitetets lokaler reserveres vha. mail.
    % XXX Komstyret: Denne må fikses hvis kommunikasjonsstyret innføres
	\item Blæsting. Det er viktig å gi blæst informasjon, de fikser også event
          på \texttt{facebook}. Også webansvarlig bør få informasjon til
          hjemmesiden.
	\item Gave til foredragsholder, for eksempel en flaske vin. 
          Taes ut fra kjelleren; før bilag og \emph{lever lapp} til
          Kjellermester. Se også \ref{fagstyret:sec:bilag}.
	\item Mandag$^\text{TM}$ kan gjerne gjennomføres etterpå, hvis
          arrangementets kjer på kveldstid.
\end{itemize}

Populærvitenskapelige foredrag er typisk det vi arrangerer oftest.
Man bør strebe etter å arrangere noe et par ganger per måned.
Det er viktig å ha noenlunde oversikt over hva som skjer av foredrag ellers
på fakultetet, så man ikke arrangerer ting samtidig som andre.

Det finnes mange potensielle foredragsholdere på universitetet, det gjelder bare
å lete.
Statlige tilsyn og foretak er også glade i å komme for å fortelle om det de
driver med.
% Hjelp til å lete kan finnes i permen på kontoret som er merket
% ``Fagstyret''. Der har vi informasjon tilgjengelig om hvem som har holdt 
% foredrag tidligere, samt kontaktinformasjon. For 2003....
I tillegg har vi en avtale med Tekna, som gir oss
hjelp til å finne kompetente folk som kan et visst tema. 
Det kan være et triks å velge ``dagsaktuelle'' temaer:
Ting som blir skrevet om i avisen osv. ---
debatter trekker alltid mange mennesker.

NB: det er \emph{helt sentralt} å ha en god \emph{ordstyrer} når man
arrangerer en debatt, ellers har det en tendens til å skli ut.

\subsection{REAL frokost}
Den vanligste pop-viten er i skrivende stund i form av REAL frokost:
Fagstyret (evt. forsterket av Arr) står opp tidlig (de er på blindern
ca.~0730) og lager frokost, som så serveres til alle som ønsker, samtidig som
det foregår en debatt eller et foredrag.
Veldig populært!
Se også \ref{fagstyret:sec:REALfrokost-oppskrift}.

\subsection{Panikkhjelp}
Fagstyret arrangerer for tiden panikkhjelp i første\-års\-emner på matematisk:
MAT1001, MAT1100, MAT-INF1100 og STK1000 i høstsemesteret, og MAT1110, MAT1001
 og STK1100 i vårsemesteret. ``Panikkhjelperene'' er som regel gruppelærere i
fagene, og personer i foreningen som har hatt fagene før.
Panikkhjelpansvarlig er ansvarlig for å gjennomføre panikkhjelp.

Matematisk institutt pleier å sponse panikkhjelpen med penger til mat til
``panikkhjelperene''.

Tidligere har det også vært arrangert turer til forskjellige steder.
Disse kan med fordel arrangeres i samarbeid med relevante arbeidsgivere, 
men man kan også dra på utstillinger,
eller f.eks.~``Senkveld'' på Teknisk Museum.

\subsection{Kontakt med næringslivet}
Å ha kontakter i næringslivet er viktig. Foreningen kan for eksempel tjene
penger på å holde bedriftspresentasjoner i kjelleren. I tillegg kan man gjøre
det mer fordelaktig for jobbsøkere å ha RF på CVen.

RF har en avtale med Tekna. I den forbindelse, har vi fått en liste med bedrifter
der Tekna er representert. Det kan da typisk være lurt å holde kontakt med
Tekna-tillitsvalgt i disse bedriftene, slik at denne personen kan hjelpe oss
med å overtale bedriften til å komme på besøk.

Det finnes mange gamle interne som nå jobber i relevante bedrifter, de kan være
nyttige når man skal ordne bedriftspresentasjoner og lignende.

% Dessverre står det ikke så mye her enda. Mye jobbing står igjen for å bygge opp
% denne delen. Nå skal også Biørnegildet 2004 arrangeres, og jeg håper at
% sponsoransvarlig Lars Warholm kan være behjelpelig med å skrive mer under
% denne subsubsection.

\section{Om sosialt}
Et av Realistforeningens viktigste formål er å bygge et sosialt miljø
mellom forskjellige realister, og det er derfor også et formål for
fagstyret; Det er viktig at det er godt samhold internt i fagstyret, og
at fagstyret er en integrert del av RF.
Alle nye fagstyremedlemmer bør oppfordres til å komme på internfester og Mandager.
I tillegg bør det arrangeres et par bare-fag sosiale arrangementer i løpet av
semesteret, som for eksempel fagstyremiddag og fag-på-fyll.
Fagstyremiddag kan forøvrig med fordel kombineres med det å skrive semesterberetning.

\section{Tips og triks}
\subsection{Fagstyrespesifikt om bilag}
\label{fagstyret:sec:bilag}
Bilag føres \emph{hver gang} penger skal innom foreningen. 
Fagstyret har fortiden følgende kontoer:
``7120 Pop.~Vit.~og fagstyreavgifter'',
``7062 REAL frokost'', og
``7073 Gave til foredragsholder''.
Sistnevnte er ment for  betale bl.a.~vin til foredragsholder,
den midterste bør være relativt åpenbar,
og den førstnevnte brukes til alt annet.
I tillegg finnes kontoen
``7060 Eksterne arrangementer'', men den er skal egentlig bare brukes av Arr.
Hvis du ikke vet hvor du finner bilagsskjemaer, kan du spørre ØU-sjef.
(Vedkommende vil mest sannsynlig også fortelle deg hvordan man fører bilag.)

\subsection{Blæsting}
For at et arrangement skal bli en suksess, er det nødvendig at det kommer
folk. For at det skal komme folk, er det nødvendig at de informeres om
arrangementet.
Blæst er eksperter på dette. De både lager og henger opp plakater, og lager
dessuten event på facebook osv.
Det er viktig å alltid holde blæst informert om kommende arrangementer.
For tiden løses dette bl.a.~med at blæstsjef er med på fagstyre-listen.
Det er allikevel viktig med direkte kommuniksajon mellom Fagsjef (eller
den som er ansvarlig for et spesifikt arrangement) og Blæstsjef.

Det kan alltid være lurt å snakke med foredragsholderen for å finne ut
hva som skal skje. Han/hun kan skrive et kort sammendrag, og kanskje komme
med bildemateriale som kan brukes på plakater/banner og nett.

Noen former for blæsting kan fagstyret stå for selv. Det er lurt å få 
arrangementet ``på plakaten'' i Universitas. Dette gjøres ved å sende
en mail til \url{til-plakaten@universitas.no}.
Det eksisterer en liste som heter \url{informasjon@rf.uio.no}
Sender man informasjon til denne lista, blir
det sendt til alle interne, samt endel eksterne som ønsker informasjon.
Listen er moderert, så enkelte ting kan ta lang tid.

\subsection{Oppskrift på et faglig arrangement}
\label{fagstyret:sec:REALfrokost-oppskrift}
Når en REAL-frokost-arrangør skal arrangere en REAL-frokost, er det en del
ting som må gjøres:
\begin{itemize}
    \item Skaffe foredragsholder eller debattanter
    \item Skaffe lokale
    \item Kontakte blæst, fikse blæsting
    \item Kontakte Regi, fikse Lyd/Lys/Etc.
    \item For REAL frokost: Handle
    \item For REAL frokost: kontakte teknisk avdeling og si at man skal
          lage mat, så de ikke dukker opp uventet og er ufine.
    \item Utføre
    \item Føre bilag (husk å be om kvitteringer når du kjøper noe)
\end{itemize}
