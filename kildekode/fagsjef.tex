% vim:fileencoding=utf-8
% -*- encoding: utf-8 -*-
%Denne fil vil trenge en
%\usepackage{amsmath,url}
%i preamble.
%
\section{Fagsjef}
Fagsjef sitter i hovedstyret, og i tillegg til i fagstyret. Utdrag fra lovene 
angående fagstyret:
\begin{quote}
§ 10. FAGSTYRET 
\begin{itemize}
	\item[a)] Fagstyret ledes av Fagsjef, og har følgende andre medlemmer: 
		Populærvitenskapelig sjef og Bedriftskontakt.
	\item[b)] Fagstyret har ansvar for foreningens faglige profil og 
		kontakt med næringslivet.
	\item[c)] Fagstyret skal fremlegge forslag til foreningens faglige 
		program for Hovedstyret.
	\item[d)] Fagstyret har ansvaret for Fagstyrets regnskap og 
		økonomistyring.
\end{itemize}
\end{quote}

\subsection{Ansvarsområder}
Fagsjefen leder fagstyret, som har ansvaret for Realistforeningens faglige
profil og kontakt med næringslivet. I tillegg til å lede fagstyret, er
fagsjefen, som regel også en av Populærvitenskapelig sjef eller Bedriftskontakt.
I motsetning til de fleste andre verv, er fagsjef en ganske fri stilling. Ting
som gjøres skjer oftest på eget initiativ.

\subsubsection{Foreningens faglige profil}
\begin{quote}
	Allerede de aller tidligste lover slo fast at foreningens formål er delt: en
	faglig og en sosial del. De tidligste møter i foreningen ble innledet
	med foredrag og debatter før de gikk over i fest og moro. Selv om den
	faglige delen i dag står noe svakere, er den en viktig del av
	foreningen.
\end{quote}
For foreningens faglige profil arrangeres det fortsatt foredrag, kurs og andre
ting.

For å arrangere noe kreves følgende:
\begin{itemize}
	\item Personer, for eksempel foredragsholdere. Dette er den mest
	krevende faktoren.
	\item Lokale. Universitetets lokaler kan i dag reserveres på
	\url{http://www.locus.uio.no}
	\item Blæsting. Det er viktig å gi blæst informasjon. Også webansvarlig
	bør få informasjon til hjemmesiden.
	\item Gave til foredragsholder. For eksempel ei flaske vin. Kan kjøpes i
	kjelleren mot kontant betaling.
	\item Mandag$^\text{TM}$ etterpå hadde vært fint.
\end{itemize}

Populærvitenskapelige foredrag er typisk det vi arrangerer oftest. Man
bør strebe etter å arrangere noe hver uke, men det er ikke nødvendig for
at medlemmene blir fornøyde. Ha det likevel som mål!

Det finnes mange potensielle foredragsholdere på universitetet, det gjelder bare
å lete. Hjelp til å lete kan finnes i permen på kontoret som er merket
«Fagstyret.» Der har vi informasjon tilgjengelig om hvem som har holdt 
foredrag tidligere, samt kontaktinformasjon.
I tillegg har vi en avtale med Tekna, som gir oss
hjelp til å finne kompetente folk som kan et visst tema. 

I skrivende stund har ikke dette skjedd enda, men vi er klar til å arrangere
oppgaveløsningskveld med pizza. Tanken er å hjelpe nyere funksjonærer med
oppgaver relevant for eksamen, så denne kvelden kan typisk arrangeres siste
måned før eksamen. Foreleserne i de forskjellige kursene synes som regel
dette er et godt tiltak, og blir gjerne med, såfremt de blir invitert i tide.
Kvelden arrangeres gjerne i kjelleren, med salg av mineralvann, og gjerne
andre ting (ikke nødvendigvis alkohol.) Etterpå blir det pizza på huset.

Tidligere har det også vært arrangert turer til forskjellige steder. For
eksempel museer og dyreparker, men også andre severdige ting.

\subsubsection{Kontakt med næringslivet}
Å ha kontakter i næringslivet er viktig. Foreningen kan for eksempel tjene
penger på å holde bedriftspresentasjoner i kjelleren. I tillegg kan man gjøre
det mer fordelaktig for jobbsøkende å nevne RF i sin CV.

RF har en avtale med Tekna. I den forbindelse, har vi fått ei liste med bedrifter
der Tekna er representert. Det kan da typisk være lurt å holde kontakt med
Tekna-tillitsvalgt i disse bedriftene, slik at denne personen kan hjelpe oss
med å overtale bedriften til å komme på besøk.

Dessverre står det ikke så mye her enda. Mye jobbing står igjen for å bygge opp
denne delen. Nå skal også Biørnegildet 2004 arrangeres, og jeg håper at
sponsoransvarlig Lars Warholm kan være behjelpelig med å skrive mer under
denne subsubsection.

\subsubsection{Blæsting}
For at et arrangement skal bli en suksess, er det nødvendig at det kommer
folk. For at det skal komme folk, er det nødvendig at de informeres om
arrangementet. Til dette formål er RFs blæstgruppe tilgjengelig. For
at da blæst skal ha noen sjans til å blæste arrangementer, er det nødvendig
at de kjenner til arrangementet. Ta tidlig kontakt med informasjonssjef!

Det kan alltid være lurt å snakke med foredragsholderen for å finne ut
hva som skal skje. Han/hun kan skrive et kort sammendrag, og kanskje komme
med bildemateriale som kan brukes på plakater og banner.

Noen former for blæsting kan fagstyret stå for selv. Det er lurt å få 
arrangementet «på plakaten» i Universitas. Hvordan man gjør dette kan leses
i «på plakaten»-siden i avisa. Det eksisterer også ei liste som heter
\url{informasjon@rf.uio.no}. Sender man informasjon til denne lista, blir
det sendt til alle interne, samt endel eksterne som ønsker informasjon.
Lista er moderert, så enkelte ting kan ta lang tid.

