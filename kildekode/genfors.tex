% vim:ft=tex:fileencoding=utf-8:tabstop=4:shiftwidth=4
% -*- encoding: utf-8 -*-

\section{Forberedelser}
Det er ikke mye som må gjøres foran en ordinær Generalforsamling, men det er
viktig at ting blir gjort.

\subsection{Arrangement}
Tidligere har man kombinert Generalforsamling med andre aktiviteter. Dette kan 
være en idé for å trekke folk. Generalforsamlingen bør blæstes godt for Interne,
men alle betalende medlemmer har adgang.

Det har vært vanlig å sette frem noen kasser Øl til de som dukker opp, og etter
Generalforsamlingen er det gjerne pizza på huset. Det er ikke sikkert man blir
settedyktig, så vent med serveringen til man er igang.

\subsubsection{Oppgavefordeling}
Det er lurt å ha kandidater til formalitetene klart på forhånd. Formannen er
tradisjonelt ordstyrer, og bør ikke ha andre oppgaver under møtet. Sekretær
kan med fordel skrive protokoll. Personer som underskriver protokoll bør være
mulig å få tak i, og bør finnes på forhånd. Dette bør helst være andre personer
enn Formann, Sekretær eller DU.

Videre: Pizza må bestilles, og øl kan kjellermester skaffe.

På forhånd er det lurt å ta kontakt med Regi, slik at det blir vist Donald-filmer
i pausen(e). Auditorium 1 i VB er vanligvis lokalet; dette må holdes av.

\subsection{Innkalling}
Det må kalles inn til Generalforsamling minst ti virkedager før. Feriedager teller
ikke som virkedag. \emph{I tillegg er det verdt å merke seg at virkedager telles på sekundet:
Er generalforsamlingen berammet til onsdag klokken 18.04, skal innkallelsen sendes ut
(hvis det ikke er ferier i veien) senest klokken 18.04 fjorten dager tidligere.} I skrivende
stund er lovene slik at man må være medlem når inkallelsen kommer for å få stemmerett, og derfor
er det lurt å kalle inn etter siste arrangement før fristen.

Det er også lurt å legge forslag til dagsorden sammen med innkallelsen. Forslag til dagsorden
må senest sendes ut fem virkedager før møtet, men det er ingen grunn til ikke å ta det med
allerede ved innkallingen. Det er viktig å merke seg at Generalforsamlingen bare kan fatte vedtak
når saken er foreslått tatt opp fem virkedager før møtet.

Innkallelsen må offentliggjøres. En epost til informasjon@rf.uio.no holder til det, men det 
bør også henges opp papirkopier på RFs tavler. Innkallelsen bør også inneholde et notat
om listen diskusjon@rf.uio.no og hvordan man melder seg på denne – hvis det varsles på 
innkallelsen, kan denne brukes til å foreslå saker for Generalforsamlingens overveielse,
inkludert lovendringer. Innkallelsen bør isåfall inneholde korrekt prosedyre for å foreslå
saker.

\subsection{Økonomi}
Hovedstyrets primæroppgave er å tenke langsiktig, og i så måte er budsjettet viktig.
Hovedstyret skal innen en uke før Generalforsamlingen komme med et forslag til budsjettrammer
som Generalforsamlingen skal vedta. Det er viktig at dette forslaget er lettleselig og godt
kommentert, for andre personer skal kunne fremme endringsforslag innen 48 timer før møtet.
Det er viktig at det finnes en reell mulighet for dette, for fremtidige planer avhenger
gjerne av budsjettet.

I tillegg må Økonomiutvalget kunne gjøre rede for RFs økonomi, og Revisjonsutvalget
må skrive sin beretning.

\subsection{Semesterberetning}
Semesterberetningen kommer gjerne skriftlig med sakspapirene. Hver gruppering skriver sin.
Få med navn på interne og hva som er gjort dette semestret, gjerne med en liten evaluering.

\section{Lovendringsforslag og andre saker}
Hvis Generalforsamlingen skal fatte vedtak i en sak, må saken foreslås tatt opp minst
fem virkedager før. Forslagsstiller må selv ta ansvar for å offentliggjøre forslaget, f.eks.
ved å sende forslaget til diskusjon@rf.uio.no hvis dette er foreslått på innkallelsen.
Hovedstyret (hs@rf.uio.no) skal ha kopi av alle forslag.

Hvis forslaget er et lovendringsforslag, er det strengere krav enn andre forslag. Et lovendringsforslag
må leveres inn med alle formuleringer: Man kan ikke bare foreslå en intensjon, og overlate til
Generalforsamlingen eller andre å formulere den. Generalforsamlingen kan ikke endre på ordlyden
i lovendringsforslag, bare vedta eller avvise forslaget.

Saker det ikke skal fattes vedtak i, kan foreslås når som helst. Men merk at disse begrensningene
er til for at ikke Generalforsamlingen skal ta for mye tid – en ordinær Generalforsamling tar 
allerede to–tre timer, og for å bli vedtaksdyktig trenger man gjerne en del mennesker som
ikke er interessert i detaljer omkring lovverk.

\subsection{Desisjonsutvalget}
Desisjonsutvalget har utvidede rettigheter til å foreslå endringsforslag. Hvis desisjonsutvalget
liker forslaget, er det vanlig praksis at det retter opp «formaliafeil» eller foreslår en annen ordlyd.
Desisjonsutvalgets muligheter gjør det også mulig å rette opp konflikter i innsendte forslag.

\section{Avstemninger}
Det er forskjellige former for avstemninger. RF bruker alminnelig flertall, absolutt flertall
og 2/3-flertall.

\subsection{Alminnelig flertall}
Alminnelig flertall har det alternativ som får flest stemmer. Alminnelig flertall er den vanligste
formen på generalforsamlingen, og er også formen som brukes ved akklamasjon.

For alminnelig flertall, holder det med én stemme så lenge ingen stemmer noe annet. Derfor kan mange
avstemninger gjennomføres ved først å forsikre seg om at ingen stemmer mot, for hvis ordstyrer stemmer
for, er denne stemmen nok.

Hvis det er like mange kandidater som verv (eller bare ett alternativ å stemme over), kan man gjennomføre
avstemningen ved akklamasjon – å klappe betyr å stemme for.

Hvis det er flere kandidater, stemmer de stemmeberettigede på like mange kandidater som det er verv.
De som har fått flest stemmer, blir da valgt.

\subsection{Absolutt flertall}
Hvis det skal velges én person til et verv, kan tre personer kreve absolutt flertall. Absolutt flertall
betyr at minst halvparten av forsamlingen må stemme for alternativet.

Skriftlig valg hvor det er krevd absolutt flertall, kan med fordel avgjøres etter preferansevalgmetoden.
I teorien betyr dette at man har flere valgomganger, og fjerner den minst populære kandidaten i hver omgang
inntil noen har absolutt flertall. Preferansevalg innebærer at kandidatene rangeres på stemmeseddelen,
og at tellekorpset teller stemmesedlene flere ganger og gir stemme til den første kandidaten som fortsatt er med
for hver seddel.

En annen metode for absolutt flertall, er «majoritetsvalg», der de to mest populære kandidatene går videre til
en ny runde hvis ingen får absolutt flertall ved første forsøk. Denne metoden brukes f.eks. i det franske presidentsvalget.

Hvis flere skal velges samtidig, kan absolutt flertall ikke benyttes.

\subsection{2/3-flertall}
Lovendringer vedtas med 2/3-flertall. Dette krever at 2/3 av forsamlingen aktivt må stemme for forslaget.
Dersom man har flere motstridende forslag, bør man derfor først finne den mest populære ved å sette to og to
forslag opp mot hverandre, og ta avstemning for vinneren.

\section{Møtets gang}
Generalforsamlingen behandler mange saker. Noen er vedtakssaker, mens andre er orienteringssaker.
I vedtakssaker \emph{må} referatet gjengi vedtaket. Dette er de typiske saker, ikke alle står i lovene.

Generalforsamlingen settes ikke før man er settedyktig, og da bæres HKH Ursus Minor inn (alle reiser seg), og
han får en øl.

\subsection{Godkjennelse av innkallelsen}
Innkallelsen kan bare godkjennes hvis Desisjonsutvalget konkluderer med at den var gyldig.
Dersom så er tilfelle, kan generalforsamlingen vedta å godkjenne. Ordstyrer bør her spørre
om noen tegner seg imot innkallelsens godkjennelse. Hvis ingen tegner seg mot, kan ordstyrer
– hvis han har stemmerett og stemmer for godkjennelse – trygt anta at dette har flertall.
Hvis noen stemmer mot, må flere stemme for.

\subsection{Godkjennelse av dagsorden}
Generalforsamlingen kan endre dagsorden frem til den er vedtatt, men det er ikke anledning til å
fremme nye vedtakssaker. Dagsorden vedtas typisk på samme måte som innkallelsen.

\subsection{Valg av ordstyrer og referent}
Ordstyrer og referent er gjerne hyret inn på forhånd, og velges som regel ved akklamasjon.

\subsection{Valg av to personer til å underskrive protokollen}
Disse er helst spurt på forhånd. Ellers bør ordstyrer foreslå passende personer fra forsamlingen.

\subsection{Valg av to personer til å plukke papirfly}
Hvis/når HKH Ursus Minor treffes av et papirfly, foreslås denne personen til papirflyplukking.

\subsection{Regnskaper}
ØU-leder får gjerne innlede denne. Regnskapet presenteres (det er fint om en kopi er delt ut) på overhead e.l.,
og forsamlingen bør få en forståelse for foreningens økonomi.

Revisjonsutvalget vil legge frem regnskaper som er revidert med tilhørende revisjonsberetning. Etter
RUs innstilling, kan generalforsamlingen med simpelt flertall godkjenne eller ikke godkjenne regnskapet.
Godkjennelse av regnskapet betyr at de ansvarlige innvilges ansvarsfrihet for regnskapet.

\subsection{Budsjettrammer}
Alle forslag til vedtak, må være offentliggjort innen generalforsamlingen. Et av dem velges med simpelt 
flertall.

\subsection{Lovendringsforslag}
Desisjonsutvalget vil gjerne kunne informere om forslagene. Et forslag må vedtas med 2/3-flertall.
Det vil ofte være megen debatt ved lovendringsforslag. Ordstyrer må her sørge for at dette ikke går over
styr. Det kan være lurt å ha hjelp til å opprettholde en taleliste. Pass på at denne følges. Hvis 
argumenter gjentas, kan taleren stanses (klubbes).

\subsection{Semesterberetninger}
Dette er en orienteringssak/diskusjonssak. Semesterberetningene bør deles ut på forhånd, og medlemmene
kan stille spørsmål til de interne. Det foretas ingen avstemning under denne saken.

\subsection{Kontigenter}
Forslag til kontigent fremmes på Generalforsamlingen. Et av forslagene velges med simpelt flertall.
Hovedstyret kan med fordel komme med et forslag i sakspapirene. Merk at det ikke finnes noe underforstått
forslag.

\subsection{Valg av tillitsvalgte}
De som stiller til valg får presentere seg, og deretter stemmes det. Det er vanlig å kreve skriftlig valg
dersom det er flere kandidater enn verv. Isåfall bør det oppnevnes et tellekorps. Under valg er det vanlig
å ta hensyn til at noen kanskje stiller til flere verv, og rekkefølgen på valgene kan justeres etter dette.

\subsection{Ymse}
Ymse kan brukes til å fortelle om viktige saker, og å spørre om detaljer. Generalforsamlingen kan ikke
fatte noe vedtak.

