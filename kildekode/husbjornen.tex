% vim:fileencoding=utf-8
% -*- encoding: utf-8 -*-
% vim:fileencoding=iso-8859-1

\section{Husbjørnen}

Denne teksten er rettet mot de som er med i Husbjørnen. Unntak er dog
den første teksten og den om trykking, som er ment for redaktøren. I
tillegg er teksten om komponering rettet mot personen som er ansvarlig
for design og komponering.

\subsection{Husbjørn-redaktør}
 Husbjørn-redaktøren har som oppgave å produsere, distribuere og stå
til ansvar for innholdet i Husbjørnen - realistenes avis. For å få til
dette er det nesten absolutt nødvendig å ha en redaksjon.

 Veldig mye av arbeidet kan gjøres gjennom dem. Dette gjelder spesielt
innsamling av materiell. Jobben inneholder mange detaljer, og det er
fort gjort å bomme på et eller annet, så vær strukturert.

 Det anbefales sterkt at man holder møter minst en gang i måneden, og
holder en klar saksliste med punkter som skal dekkes, slik at ikke
ting blir glemt. Regn alltid med at ett eller annet blir glemt.

 Sett fristene for innlevering av materiell til minst 14 dager før
bladet går til trykkeriet. Dette hjelper både for personer som er sene
med artiklene sine, og gir tid til personen som designer avisen til å
komponere alt sammen.

 Erfaring viser at hvis tekster blir for lange, så gidder ikke folk å
lese dem. Hold informasjon ned til maksimalt to sider inklusivt
bilder. For A5 formatet, så betyr det en absolutt øvre grense på rundt
1000 ord. Det er da oversiktelig hvis de fleste tekstene er på opptil
rundt 500 ord.

 For å ordne med alt som har med terminalstuer å gjøre, så tar man
kontakt med ansvarlig for Drift: Stein Michael Storleer på
michael@matnat.uio.no. Han kan også gi kontaktinformasjon til de
ansvarlige for de forkjellige terminalstuene. Hvis man skal ha kontakt
med alle stueansvarlige samtidig, kan man eventuelt ta kontakt på
stueansvarlige@matnat.uio.no.

 For å ordne med distribusjon utenom termstuer, så tar man kontakt med
ansvarlig for Teknisk. Dette er for tiden Astri Ottesen med e-post
astri.ottesen@admin.uio.no.

\subsection{Innsamling av materialet}
 Hva som er med er opp til redaktør og redaksjonen. Man kan ikke si at
en måte er bedre enn en annen. Men per dags dato er dette oppsettet
for hva vi gjør:

\subsubsection{Intervju}
 For å lage et godt intervju hjelper det å ha gode spørsmål. For
personlige spørsmål, snakk med mennesker som kjenner ham/henne. For
upersonlige spørsmål snakk med en som kan temaet intervjuobjektet blir
intervjuet for.

 Et profesjonelt intervju gjøres med en diktafon. Hvis dette ikke er
så viktig, anbefaler jeg at man samles foran en Pc og bytter på å
skrive med objektet. Eventuelt kan man til nøds også skrive intervjuet
fortløpende. Dette går dog tregt.

 Få tak i fotografier av intervjuobjektet.

\subsubsection{Pop.vit}
 Forutsagt at man vet hva man ønsker en populærvitenskapelig artikkel
om, så er det bare å ta kontakt med forskjellige professorer på
universitet og spørre. Mange av dem er entusiastiske om tanken om å få
publisert tanker om hva de driver med.

 Få tak i fotografier forfatteren av artikkelen.

\subsubsection{Husbjørnen for $n$ år siden}
 Interessante artikler fra Husbjørnens arkiv. For å få til dette må
man lete i arkivet etter en artikkel man synes er interessant. Dette
må da skrives for hånd inn på en datamaskin siden det ikke finnes
digitale kopier. Redaktør skal ha oversikt over arkivet.

 Eventuelle bilder og ingresser må også ordnes.

\subsubsection{Svada Hall of Fame}
 Sitater fra e-postlisten svada i RF. Dette ordnes typisk gjennom en
fast person som er på denne listen. Hvis man ikke har en, så spør
rundt. Dette er ikke en stor oppgave, så flere er nok villige. Alt man
trenger for å gjennomføre denne spalten er å purre på for sitater,
slik at de kommer inn i tide, og å få bekreftet at man har tillatelser
til å sitere personen på sitatet deres.

\subsubsection{Blindernekskursjonene}
 Vurderinger av forskjellige tilbud rundt omkring på Blindern. Dette
har vært lettest å kombinere som et internarrangement for redaksjonen.
De som blir med får prøve ting gratis, og alle bidrar med idéer til
innhold. Denne ordningen har lykkes ekstremt bra hittil.

\subsubsection{DJ-panelet}
 Diskusjon mellom DJ-ene i RF om forskjellige spørsmål man stiller
dem. Denne spalten ordnes ved å stille et eller annet musikkrelevant
spørsmål direkte på Regis e-postliste: regi@rf.uio.no. Lag et
sammendrag, korriger navnene og send det tilbake til listen for
godkjenning. Typisk gjør folkene på listen veldig mye av jobben selv.

 Legg ved en ingress.

\subsubsection{Formannen har ordet}
 Denne spalten skriver seg selv så lenge man klarer å overtale Formann
til å skrive. Dette er ikke nødvendigvis en enkel oppgave.

 Eventuelle fotografier må også ordnes.

\subsubsection{Dikt}
 Hvis man ikke har et godt dikt på lur, eller kjenner noen som har
det, så kan man igjen ty til Husbjørnens arkiv. Mange av de gamle
avisene har dikt i seg som er både fagrettet og interessante (hvis man
liker den slags).

 Legg ved relevante bilder/fotografier hvis man klarer å skaffe noen.

\subsubsection{Konkurranse}
 Denne står veldig åpen. Så lenge man er villig til å gjennomføre en
bekreftelse av hva enn det er det bes om, så er det få grenser for
spørsmål. Lag gjerne en tekst rundt for å krydre det hele. Husk at det
må være med en premie, en måte å sende inn løsningsforslag til
konkurransen og en klar definisjon av hva det blir spurt om.

 Eventuelle bilder og/eller fotografier må også ordnes.

\subsubsection{Program}
 Denne burde være så oppdatert som mulig. Hold jevnlig kontakt med
Arrangementsjef om dette.

 Lag gjerne noen små tekster til en del relevante arrangementer, slik
at programmet blir mer utfyllende.

 Ta gjerne kontakt med instituttforeninger på matnat:
\verb|instituttforeningene@matnat.uio.no|

\subsubsection{Leder}
 Denne skrives hovedsakelig av Redaktøren, hvis ikke redaktøren sender
oppgaven videre. Den kan brukes som en lufteventil for uttrykke sin
frustrasjon over hvordan ting er.

\subsubsection{Tegneserie}
 Dette er et foretak som krever to elementer: En skribent som lager
plot og dialog, og en tegner som komponerer sammen. (Dette kan være
samme person.) Hvis man ikke har noen til å utføres disse to oppgavene
bra nok, så vil tegneserien falle sammen. Så ikke ha for høye
ambisjoner om innholdet i tegneserien, hvis den ikke har motivasjonen
i orden.

\subsubsection{Ymse}
 Ideer ligger og virrer rundt omkring overalt. For å få realisert dem,
så trenger man bare å få samlet en konkret vinkling, og å spørre rundt
om noen kunne tenke seg å skrive denne. Så lenge oppgaven er konkret
nok og den virker interessant, så er det flere i foreningen som kan
fungere som ypperlige frilansskribenter. Dessuten skal man også alltid
vurdere om en tekst kan konverteres til en spalte slik at man får mer
ut av den.

 Husk å legge ved ekstra informasjon som fotografier og/eller bilder
hvis det passer.

\subsubsection{Forsidetegning}
 Dette er typisk en oppgave man venter med til rundt deadline før man
starter på. Det er ikke før da man har nok stoff til å si hva du har
lyst til å ha på forsiden. Enkleste løsningen er å be noen om å tegne
noe som samsvarer med en av tekstene på innsiden. En bjørn er dog en
klassiker som sjeldent slår feil.

\subsection{De store OBS-ene}
 Noen av de største feilene som kan oppstå i en avis, er at man bruker
tekster som enten er feil, eller som man ikke har tillatelse til å
bruke. Førstnevnte er ikke et veldig stort problem, siden Husbjørnen
sjeldent publiserer fakta. Sistnevnte derimot skal man være litt
forsiktig med.

 Hvis man skal ha et fotografi av en person, husk å be om tillatelse
fra personen før man bruker det.

 Hvis man skal sitere noe, husk å spørre om lov, og å markere
hvem/hvor det kommer fra i teksten på en eller annen måte.

 Å vite om et arrangement er offentlig eller internt kan ofte være
vanskelig å bedømme. Spør alltid arrangøren om lov, hvis man skal
publisere oppføringen av et arrangement.

 Mange bilder på nett kan det være farlig å kopiere av. Den beste
måten å unngå det på er å få en tegner til å tegne en kopi. Da er man
på en tryggere side av loven. Unntaket er selvfølgelig hvis man har
bedt rettighetshaverne om lov først.

 Skal man referere til et navn eller en e-postadresse, be om lov.

 Det finnes en e-postliste som heter korrrektur@rf.uio.no. Den
e-postlisten er dedikert til å korrekturlese alle tekster som sendes i
deres retning. Jeg advarer mot å bruke listen til personlige formål,
for å ikke irritere de som står på listen for mye. Hvis det finnes en
eneste mistanke om at en tekst på en eller annen måte har bruk for
korrektur, send det til denne listen. Klipp og lim teksten som skal
korrekturleses rett inn i en e-post og still din forespørsel på toppen
av teksten. De er veldig lite glad i vedlegg.

 Når en tekst er ferdig, skal den enten til redaktør eller til
designer. Dette kommer an på om redaktør ønsker å overse ting før det
publiseres eller ikke. Igjen skal teksten være i formen av en e-post
uten vedlegg.

 Bilder skal ikke sendes som vedlegg. De skal enten overføres direkte
til enten redaktør eller designer, eller gjennom en link rettet mot
hvor filen er lagret på nettet.

\subsection{Komponering}
 For å komponere avisen brukes programmet InDesign. Denne finnes i
flere versjoner, og det kan være farlig om to versjoner er
inkompatible, og flere enn en skal ta en titt på samme utgave.
Programmet er relativt intuitivt, og støttes lett opp mot Photoshop.

 Hvordan avisen skal se ut er mer eller mindre opp til designeren. Det
er dog viktig å ha i bakhodet at dette er avisen til
Realistforeningen, og at man ikke skal fravike for mye fra tema og
logoer. Typisk skal det alltid være en del bjørner med i bladet.

\subsection{Trykking}
 Trykking skjer typisk gjennom CopyCat. Ved å sende en e-post til
nydalen@copycat.no så er mesteparten av jobben gjort. Introduser deg
selv, og be om et prisoverslag over trykkejobben du ønsker å få
utført. De fører ikke prislister, siden kalkuleringen av trykking er
en noe komplisert affære. Dessuten kan man diskutere prisen noe, hvis
man ber om en veldig stor jobb. Derimot kan man be om flere
prisoverslag om gangen. Pass på at de regner prisen uten moms. De
trenger å vite:

\begin{itemize}
 \item Format: Typisk A4 eller A5
 \item Vekt: Standard ark er på 80g eller 100g. Mer eksklusive får man ved
100g glanset og 200g glanset.
 \item Farge: Hvis man skal ha noen av arkene i en annen farge, eller man
ønsker fargesider, så må det spesifiseres.
 \item Antall: Hvor mange eksemplarer som skal trykkes.
\end{itemize}

 Per dags dato kjører vi A5, 100g glanset utside, 80g innmat og 400 eksemplarer.

 Hvis man er usikker på om hvordan ting vil se ut i et format man ikke
har sett på så kan man be om å få se på prøveeksemplarer. Dette koster
ikke noe.

 Når man skal sende ting til trykkeriet, så er det bare å sende en
e-post med avisen i form av en PDF-fil vedlagt. Eventuelt kan man
sende en link til et sted hvor man har publisert filen på nett.

 Hvis man ønsker å vite hvor lang tid det tar å trykke, så er det lite
som er i veien for å spørre. Typisk tar det en dag.

\subsection{Distribusjon}
 Når avisene er ferdig trykket, så kan de plukkes opp hos Copycat.
Dette ligger i Nydalen, på venstre side av BI bygget sett fra Nydalen
t-banestasjon. Det tar ca. to minutter å gå fra stasjonen.

 Ordningen per i dag er at det alltid er to personer som går sammen.
Dette gjøres av to grunner. Den ene grunnen er at det blir mer sosialt
å gjøre jobben. Den andre grunnen er at hvis bare en person vet hva
som skal gjøres, vil den andre også lære, og kan spre sin kunnskap
videre.

 Per i dag så har Husbjørnen et gitt sett med steder hvor den kan
plasseres. Man kan ignorere tillatelsene som er satt, men man
risikerer at bladet kan bli fjernet. Flere steder er ikke helt i orden
heller, siden vi trenger avisstand for at ting skal være helt korrekt.
På sikt er det målet. Inntil videre kan man prøve snike avisen inn ved
siden av andre magasiner hvor det er plass.

 Budrunden tar drøye to timer, og etter stengetid så krever det også
ofte tilgangskort. Pass på at dette er i orden før man begynner å
distribuere. Husk også å samle opp alle gamle blader. Disse skal
redaktøren ha.

 Vi har spredt ting ut på stands og terminalstuer rundt omkring på
Blindern. De med en «*» ved mangler per dags dato stand.

\begin{itemize}
 \item Biologi. Ved inngangen nærmest PO
 \item Biologi. I utskriftshyllen i terminalstuen plassert lengst unna PO
 \item PO. I utskriftshyllen ved printeren ved inngangen.
 \item Sophus Lie. Ved inngangen
 \item Nils Henrik Abels hus. I utskriftshyllene i terminalstuen i kjelleren
 \item * Fysikk. Ved pendelen
 \item Fysikk. I utskriftshyllen i terminalstuen i fjerde etasje.
 \item * Kjemi. Utenfor terminalstuen. (Foreløpig i terminalstuen på
rotebordet ved inngangen)
 \item * IFI. Ved inngangen av Terminalstuen i første etasje. (foreløpig
på rotebordet)
 \item * IFI. Ved inngangen til kantinen i andre etasje.
\end{itemize}

 På IFI har vi fått klar beskjed om at «det er bedre å be om
tilgivelse enn tillatelse». Dvs. at vi har ganske frie hender der.

