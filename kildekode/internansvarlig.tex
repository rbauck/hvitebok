% vim:fileencoding=utf-8
% -*- encoding: utf-8 -*-
\section{Internansvarlig}

\subsection{Mandat}
Internansvarlig har som oppgave å legge til rette for et godt sosialt miljø
i foreningen, og få folk til å trives. Vervet innebærer å gjøre seg kjent
med alle aktive interne medlemmer, med spesiell vekt på nye medlemmer,
og sørge for at de trives gjennom hele semesteret.

Dette innebærer ikke bare at de trives sosialt, men også at de mestrer sine
egne verv. Internansvarlig skal derfor også holde et øye åpent for om noen
trenger mer opplæring eller hjelp til arbeidsoppgavene, og i så fall ta
initiativ til at de får dette.

Internansvarlig har taushetsplikt dersom interne som tar kontakt om et
hvilket som helst tema ønsker dette.

Internansvarlig har ansvar for å oppdatere denne teksten med nye erfaringer etter fullført verv.


\subsection{Typiske oppgaver}
En viktig praktisk oppgave vil være å holde oppdaterte lister over interne til
enhver tid, med fullt navn, telefonnummer, epostadresse og postadresse.

Det er viktig at internansvarlig holder god kontakt med kjellerstyret,
som har de fleste funksjonærene under seg.

Ellers er internansvarlig er et ganske fritt verv, hvor oppgavene må tilpasses
behovet i foreningen og enkeltpersonene. Det er viktig at der andre ser
på hvordan vervet skjøttes, ser internansvarlig på hvordan personen bak
vervet trives. Typiske oppgaver vil være å jobbe for å opprettholde
Mandagstradisjonen, sørge for at internfestene ikke blir glemt,
og være med og arrangere andre internarrangementer som hyttetur og
Danmarkstur. Det er spesielt viktig å invitere nye interne med på
internarrangementene, slik at de trives og blir i foreningen.

SiO Læringsmiljø holder på med å utvikle et hefte om utbrenthet.
Det vil være naturlig at Internansvarlig henter råd derfra og
jobber for å forebygge dette i foreningen.

\subsection{Tips og triks}
Internansvarlig er foreløpig et verv som ikke har vært bemannet
på en stund, og det er derfor opp til den som tar vervet å prøve
seg frem og gjøre seg erfaringer.

Det er i hvert fall viktig at internansvarlig er tilstede ved
kranførerkurs og andre rekrutteringstiltak, for slik å bli kjent
med nye interne. Det kan også være en fordel om internansvarlig
hilser på alle nye funksjonærer i baren, kafeen og blæst og
spør hvordan de trives.

\subsubsection{Om informasjon}
E-post er en viktig informasjonskanal, sørg for å bruke denne
aktivt til å informere om alt som skjer i god tid. Men e-post er
et tveegget sverd, da mange ikke leser denne i det hele tatt eller særlig
ofte. Det er derfor viktig å bruke andre informasjonskanaler som
jungeltelegrafen, SMS (det jobbes for tiden med å få til en avtale om bruk
av SMS-server til masseutsendelser) og telefon. Og ikke minst, snakk
med så mange som mulig personlig. Dette er spesielt viktig ovenfor
nye interne, som ikke kjenner miljøet og tradisjonene så godt.
