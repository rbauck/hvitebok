% vim:fileencoding=utf-8
% -*- encoding: utf-8 -*-
\section{Kommunikasjon}
Den vanligste årsaken til at noe ikke går som planlagt, er
kommunikasjonsproblemer. Noen enkle tips for å unngå dette:

En kort samtale ansikt til ansikt eller over telefon er ofte raskere og klarere enn tekstmeldinger eller e-post.

\subsection{E-post}
Dette kan se ut som motsetninger, men det reflekterer at bruk av e-post
er et tveegget sverd. Brukt på riktig måte er det veldig effektivt,
brukt på feil måte skaper det kaos.
\begin{itemize}
\item Bruk e-post aktivt, både personlig og lister. Les e-posten din daglig. Dette kan lette arbeidet utrolig mye i mange tilfeller.
\item Ikke skriv for mye til listene, det undergraver effektiviteten fordi mange da slutter å lese så nøye det som kommer.
\item Aldri anta at en e-post har blir lest. Hvis den er viktig har den sannsynligvis ikke det.
\item E-post fungerer generelt ikke til hastemeldinger.
\item Bruk riktig liste til meldingene dine. Svada hører ikke hjemme på seriøse lister.
\item Visste du at du kan sortere e-post fra lister automatisk i forskjellige mapper? Anbefales! Spør noen om hjelp hvis du ikke vet hvordan.
\end{itemize}


\section{Ansvar}
Jo lenger opp i rekkene du kommer, jo oftere må du avstå fra å gjøre enkelt oppgaver for å ikke slite deg ut. Husk at ansvar for at noe blir gjennomført ikke betyr at du må gjøre det selv. Husk også at ansvar for noe også innebærer å sørge for å få tak i folk som kan gjøre jobben.

\section{Tidsbruk}
Ingen ønsker at du skal brenne lyset i begge ender og ende opp utslitt. Det er fort gjort at studier, foreningsarbeid og fritid blandes litt for mye, og det kan bli slitsomt. Det er derfor viktig å sette grenser. Hvordan du organiserer deg er til syvende og sist kun opp til deg selv.
