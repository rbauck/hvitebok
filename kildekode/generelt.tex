\section{Kommunikasjon}
Den vanligste �rsaken til at noe ikke g�r som planlagt, er
kommunikasjonsproblemer. Noen enkle tips for � unng� dette:

En kort samtale ansikt til ansikt eller over telefon er ofte raskere og klarere enn tekstmeldinger eller e-post.

\subsection{E-post}
Dette kan se ut som motsetninger, men det reflekterer at bruk av e-post
er et tveegget sverd. Brukt p� riktig m�te er det veldig effektivt,
brukt p� feil m�te skaper det kaos.
\begin{itemize}
\item Bruk e-post aktivt, b�de personlig og lister. Les e-posten din daglig. Dette kan lette arbeidet utrolig mye i mange tilfeller.
\item Ikke skriv for mye til listene, det undergraver effektiviteten fordi mange da slutter � lese s� n�ye det som kommer.
\item Aldri anta at en e-post har blir lest. Hvis den er viktig har den sannsynligvis ikke det.
\item E-post fungerer generelt ikke til hastemeldinger.
\item Bruk riktig liste til meldingene dine. Svada h�rer ikke hjemme p� seri�se lister.
\item Visste du at du kan sortere e-post fra lister automatisk i forskjellige mapper? Anbefales! Sp�r noen om hjelp hvis du ikke vet hvordan.
\end{itemize}


\section{Ansvar}
Jo lenger opp i rekkene du kommer, jo oftere m� du avst� fra � gj�re enkelt oppgaver for � ikke slite deg ut. Husk at ansvar for at noe blir gjennomf�rt ikke betyr at du m� gj�re det selv. Husk ogs� at ansvar for noe ogs� inneb�rer � s�rge for � f� tak i folk som kan gj�re jobben.

\section{Tidsbruk}
Ingen �nsker at du skal brenne lyset i begge ender og ende opp utslitt. Det er fort gjort at studier, foreningsarbeid og fritid blandes litt for mye, og det kan bli slitsomt. Det er derfor viktig � sette grenser. Hvordan du organiserer deg er til syvende og sist kun opp til deg selv.
