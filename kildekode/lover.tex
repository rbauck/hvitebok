% vim:fileencoding=utf-8
% -*- encoding: utf-8 -*-
\section*{FORMÅL OG ORGANISASJON}


\subsection*{§ 1}

Realistforeningen er fakultetsforening ved Det
matematisk-naturvitenskapelige fakultet, Universitetet i Oslo.


\subsection*{§ 2}

Realistforeningens formål er å samle foreningens medlemmer til å
drøfte spørsmål av almen og faglig art, å øke kameratskapet mellom
studentene og styrke miljøet ved fakultetet ved selskapelige samvær,
ekskursjoner og andre tiltak.


\subsection*{§ 3}

Realistforeningen har tre kategorier medlemmer: 

\begin{itemize}
\item [a)] Semesterbetalende medlemmer 
\item [b)] Livsvarige medlemmer 
\item [c)] Æresmedlemmer 
\end{itemize}

Semesterbetalende medlemmer og livsvarige medlemmer kan de bli som
studerer, har studert eller underviser ved Det
matematisk-naturvitenskapelige fakultet. Medlemskontingent for
livsvarige medlemmer er 10 x ordinær semesterkontingent.
Æresmedlemmer utnevnes på generalforsamling med 3/4 flertall. Alle
kategorier medlemmer har samme rettigheter.


\subsection*{§ 4}

Hovedstyret avgjør tvilsspørsmål om hvem som er berettiget til
medlemsskap.  Hovedstyret har anledning til å ekskludere et medlem,
som forser seg mot foreningens lover eller viser utillatelig oppførsel
på foreningens arrangementer.


\subsection*{§ 5}

Går en generalforsamling inn for å oppløse foreningen, skal
desisjonsutvalget tidligst tre uker og senest fire måneder senere
innkalle til ekstraordinær generalforsamling hvor saken skal behandles
på nytt. Dersom foreningen oppløses, disponerer generalforsamlingen
over foreningens aktiva.


\subsection*{§ 6}

Realistforeningen har følgende faste styrer og utvalg: 

\begin{itemize}
\item[a)] Hovedstyret 

\item[b)] Arrangementstyret 

\item[c)] Fagstyret

\item[d)] Blæstgruppa

\item[e)] Kjellerstyret

\item[f)] Økonomiutvalget 

\item[g)] Desisjonsutvalget 

\item[h)] Revisjonsutvalget 

\item[i)] Regi 

\item[j)] Vaktgruppa 

\item[k)] Husbjørnredaksjonen
\end{itemize}

I tillegg kan foreningen ha en eller flere komiteer til å ta seg av
oppgaver som faller utenfor de faste styrer og utvalgs virksomhet.



\section*{STYRER OG UTVALG}


\subsection*{§ 7}

Generalforsamlingen vedtar budsjettrammer alle styrer, utvalg og
komiteer skal følge.  Detaljerte budsjettforslag skal utarbeides av
Realistforeningens styrer, utvalg og komiteer og leveres Hovedstyret
innen en måned etter generalforsamlingen.  Alle budsjetter skal være i
tråd med budsjettrammene og godkjennes av Hovedstyret.  Overskridelser
av budsjettrammer i de respektive styrer, utvalg og komiteer behandles
av Økonomiutvalget.

\subsection*{§ 8}

Alle faste styrer og utvalg nevnt i §6, tillitsvalgte valgt av Generalforsamlingen og innehavere av verv
oppnevnt av Hovedstyret ihht. §9e), har ansvar for å lage og oppdatere erfaringsprotokoller
for opplæring av etterfølgende
innehavere av vervet og funksjonærer tilknyttet
styret ihht. §9f). Alle personer omfattet av første ledd har ansvar for at påtroppende vervinnehaver får den
nødvendige opplæring for å inneha sitt verv.
I denne sammenheng skal også disse tekster samt RFs lover gjennomgås.

\subsection*{§ 9 HOVEDSTYRET}

\begin{itemize}

\item[a)] Hovedstyret har følgende stemmeberettigede medlemmer: Formann,
Sekretær, Arrangementssjef, BlæStsjef, Fagsjef, Kjellermester, Regiformann
og Økonomiutvalgets leder.

\item[b)] Hovedstyrets stemmeberettigede velges på generalforsamling for ett
semester. Unntaket er Regiformann som velges ihht. § 17 a).  Alle medlemmene velges ved særskilt valg og må ha vært med
i minst et av Realistforeningens styrer og utvalg nevnt i § 6 minst
ett semester tidligere eller ha innehatt verv nevnt i § 9 e).

\item[c)] Sekretæren er Formannens stedfortreder og fører referat fra
Hovedstyrets møter.

\item[d)] Hovedstyrets oppgave er å koordinere langsiktig virksomhet,
godkjenne program, vedta budsjetter, utarbeide forslag til budsjettrammer 
for Generalforsamlingen og representere Realistforeningen utad.

\item[e)] Hovedstyret oppnevner alle verv. Som verv er å regne:
Skjenkemester, Utlånsansvarlig, Vaktmester, Popvitsjef, Bedriftskontakt,
Donaldsjef, Vaktgruppesjef, Arrangementsmester, Internansvarlig,
IT-ansvarlig og redaktør for Husbjørnen. Hovedstyret kan oppnevne
andre verv etter behov.

\item[f)] Hovedstyret kan bestemme at andre styrer, utvalg og komiteer selv kan
knytte til seg inntil et bestemt antall funksjonærer for å utføre
nærmere bestemte oppgaver.  Funksjonærer blir regnet som medlemmer
av gjeldende styrer og utvalg, dog uten stemmerett.

\item[g)] Innkalling til Hovedstyremøte med foreløpig dagsorden skal være
skriftlig og offentliggjort minst tre dager før møtet, med mindre det
er spesielle tungtveiende hensyn. Innkallingen skal offentliggjøres på
et lett tilgjengelig sted som Hovedstyret bestemmer. Alle RFs
medlemmer kan foreslå saker til dagsorden frem til møtet starter.
Hovedstyret vedtar endelig dagsorden etter godkjenning av innkalling,
og etter det kan bare et enstemmig Hovedstyre endre dagsorden.

\item[h)] Hovedstyret er vedtaksdyktig når der er korrekt innkalt til møtet
og minst 5 av representantene med stemmerett er til stede og saken
vedtas med alminnelig flertall. Ved stemmelikhet teller formannens
stemme dobbelt. Det skal føres møteprotokoll fra hvert av møtene som
godkjennes ved påfølgende hovedstyremøte eller etter at alle
tilstedeværende representanter med stemmerett i Hovedstyret har
mottatt og godkjent protokollen.

\item[i)] Hovedstyret skal innen en måned etter generalforsamlingen vedta en
instruks som spesifiserer arbeidsoppgavene og ansvarsområdene for
foreningens styrer, utvalg og verv.

\end{itemize}


\subsection*{§ 10 ARRANGEMENTSSTYRET}

\begin{itemize}
\item[a)] Arrangementsstyret ledes av Arrangementssjef, og har følgende andre
medlemmer: Regiformann, BlæStsjef, Vaktgruppesjef,
Arrangementsmestere og Kjellermester.

\item[b)] Arrangementsstyret er ansvarlig for planlegging, koordinering og 
gjennomføring av arrangement av sosial karakter. 

\item[c)] Arrangementsstyret skal fremlegge forslag til foreningens sosiale
program for Hovedstyret.

\item[d)] Arrangementsstyret har ansvaret for å rapportere regnskapsrelevant informasjon til Økonomiutvalget og overholde vedtatte budsjetter.
\end{itemize}


\subsection*{§ 11 FAGSTYRET}

\begin{itemize}
\item[a)] Fagstyret ledes av Fagsjef, og har følgende andre medlemmer:
Populærvitenskapelig sjef og Bedriftskontakt.

\item[b)] Fagstyret har ansvar for foreningens faglige profil og kontakt med
næringslivet.

\item[c)] Fagstyret skal fremlegge forslag til foreningens faglige program
for Hovedstyret.

\item[d)] Fagstyret har ansvaret for å rapportere regnskapsrelevant informasjon til Økonomiutvalget og overholde vedtatte budsjetter.
\end{itemize}


\subsection*{§ 12 BLÆSTGRUPPA}

\begin{itemize}
\item[a)] Blæstgruppa ledes av Blæstsjef, og kan knytte til seg så
  mange medlemmer som Blæstsjef finner ønskelig
 
\item[b)] Blæstgruppa har ansvar for å markedsføre alle foreningens
interne og eksterne arrangementer, drive rekruttering og vedlikeholde
foreningens IT-utstyr og nettsider.

\item[c)] Blæstgruppa har ansvaret for å rapportere regnskapsrelevant informasjon til Økonomiutvalget og overholde vedtatte budsjetter.
\end{itemize}


\subsection*{§ 13 KJELLERSTYRET}

\begin{itemize}
\item[a)] Kjellerstyret ledes av Kjellermester, og har følgende andre
medlemmer: Kafesjef, Utlånsansvarlig, Skjenkemestere, Vaktmester.

\item[b)] Kjellerstyret har ansvaret for den daglige drift av
RF-kjellern. Dette omfatter også utlån og vedlikehold av lokalene.

\item[c)] Kjellerstyret har ansvar for å rapportere regnskapsrelevant informasjon til Økonomiutvalget og overholde vedtatte budsjetter.
\end{itemize}


\subsection*{§ 14 ØKONOMIUTVALGET}

\begin{itemize}
\item[a)] Økonomiutvalget har fem medlemmer. Det velges to
  ordinære medlemmer på den
ordinære generalforsamling i hvert semester, og funksjonstiden er to
semestre.  Økonomiutvalgets leder blir hvert semester valgt av
Generalforsamlingen blant utvalgets fire ordinære medlemmer. I
  tillegg velges Forretningsfører for Regi i vårsemesteret, og
  sitter i to semstre. 

\item[b)] Økonomiutvalgets medlemmer kan ikke samtidig være medlemmer av noen
andre av de faste styrer, utvalg eller komiteer nevnt i §6, eller
medlemmer av Bjørnegildestyret. Unntaket er Økonomiutvalgets leder,
som er medlem i Hovedstyret, og Forretningsfører for Regi, som er
medlem i Regi.

\item[c)] Økonomiutvalget har ansvaret for Realistforeningens regnskap og
for å lære opp alle styrer, utvalg og komiteer i økonomistyring.
Økonomiutvalget kan pålegge styrer, utvalg og komiteer å føre sine
egne regnskap, men fører ellers alle regnskap.  Økonomiutvalget skal
også kontrollere at foreningens budsjett blir fulgt.  I tilfelle
budsjettsprekk, skal det aktuelle styret, det aktuelle utvalget
eller den aktuelle komiteen samt Hovedstyret, Revisjonsutvalget og
Desisjonsutvalget informeres.

\item[d)] Økonomiutvalget skal avholde konstituerende møte innen 10 virkedager
etter nyvalg. Her velger utvalget en sekretær som skal føre protokoll
over alle møter.  Det konstituerende møtet skal innkalles av lederen i
det fungerende Økonomiutvalget og samtlige medlemmer av dette
innkalles.  Det sittende Økonomiutvalg fører regnskapene ut den 
inneværende periode.

\item[e)] Revisjonsutvalget og ett medlem av Hovedstyret, i tillegg til
Økonomiutvalgets leder, har møte-, tale-, og forslagsrett på
Økonomiutvalgets møter. Utvalget kan pålegge medlemmer av styrer og
komiteer å møte ved behandlingen av bestemte saker.

\item[f)] Økonomiutvalget kan bare fatte vedtak i møte når det er minst tre
medlemmer tilstede. For gyldig vedtak kreves det at minst tre
medlemmer har stemt for forslaget. Ved stemmelikhet teller leders
stemme dobbelt.

\item[g)] Økonomiutvalget skal utarbeide forskrifter som kan lette kontrollen
med regnskapene.

\item[h)] Økonomiutvalget overtar driften av Realistforeningen inntil nytt
Hovedstyre er valgt dersom det sittende ikke kan funksjonere. Ingen
utbetalinger, med unntak av utestående fordringer, skal skje før en
generalforsamling er avholdt.
\end{itemize}
 

\subsection*{§ 15 DESISJONSUTVALGET}

\begin{itemize}
\item[a)] Desisjonsutvalget har tre medlemmer: Det velges ett medlem på den
ordinære generalforsamlingen i hvert semester, og funksjonstiden er
tre semestre.

\item[b)] Valgbare er alle som har hatt valgte verv som nevnt i §21 l).

\item[c)] Desisjonsutvalgets medlemmer kan ikke samtidig være medlemmer av
noen andre av de faste styrer, utvalg eller komiteer nevnt i §6, eller
verv nevnt i §9e eller medlemmer av Bjørnegildestyret.

\item[d)] Desisjonsutvalget har den endelige avgjørelse i tvilsspørsmål om
tolkning av lovene. Utvalget kan også fatte vedtak og gi regler i
situasjoner hvor lovene måtte vise seg å være utilstrekkelige. Ethvert
medlem av Realistforeningen har rett til å innanke for
Desisjonsutvalget vedtak hvor det kan være tvil om lovligheten.

\item[e)] Ved mistanke om misligheter kan Desisjonsutvalget suspendere
medlemmer av styrer, utvalg og komiteer. Suspensjonen kan omfatte et
organ i sin helhet, selv om det ikke foreligger konkret mistanke mot
hvert enkelt medlem.  I tilfelle suspensjon er foretatt skal
Desisjonsutvalget straks sørge for at det blir innkalt til
ekstraordinær generalforsamling der mistillitsforslag behandles og
nyvalg eventuelt avholdes.

\item[f)] Desisjonsutvalget har ansvar for at Realistforeningens arkiv til
enhver tid er i orden.
        
\item[g)] Desisjonsutvalget har møte- og talerett i alle foreningens organer. 

\item[h)] Desisjonsutvalget har ansvar for at lovtekstene oppdateres og er tilgjengelige.
\end{itemize}


\subsection*{§ 16 REVISJONSUTVALGET}

\begin{itemize}
\item[a)] Minst ett medlem til Revisjonsutvalget velges på generalforsamling
hvert semester, slik at Revisjonsutvalget til enhver tid har tre
medlemmer. Funksjonstiden er to semestre.

\item[b)] Revisjonsutvalgets medlemmer kan ikke samtidig være medlemmer av
Hovedstyret, Økonomiutvalget, Desisjonsutvalget eller Bjørnegildestyret 
eller inneha vervet som Forretningsfører i Regi eller verv som
medfører stemmerett i Kjellerstyret eller Arrangementstyret, eller ha
vært medlem av Økonomiutvalget foregående to semestre. To medlemmer av et 
styre, en komité eller et utvalg nevnt i §6 kan ikke samtidig delta i 
revideringen av et regnskap.

\item[c)] Revisjonsutvalgets oppgave er å revidere Realistforeningens
regnskaper. Minst to av revisjonsutvalgets medlemmer må delta i
revideringen av et regnskap.

\item[d)] Alle regnskaper skal være innlevert senest tre uker før
generalforsamling påfølgende semester. Blir ikke regnskapene godkjent
på generalforsamlingen, skal Økonomiutvalget inndra alle bilag og
fullføre regnskapet. Det kan gis dispensasjon til avvik fra dette
punkt av Hovedstyret i samarbeid med Revisjons- og Økonomiutvalget.

\item[e)] På Generalforsamlingen skal Revisjonsutvalget legge fram
revisjonsberetningen, som kan være skrevet av Revisjonsutvalget selv
eller, hvis Hovedstyret finner det nødvendig, en av Kredittilsynet
godkjent revisor.  Revisjonsutvalget har ansvar for å opplyse
Generalforsamlingen om eventuelle budsjettoverskridelser.
\end{itemize}


\subsection*{§ 17 REGI}

\begin{itemize}
\item[a)] Regiformann og Forretningsfører velges på generalforsamling for ett
år; Regiformann om høsten, Forretningsfører om våren.  Øvrige
medlemmer godkjennes av Hovedstyret etter innstilling fra Regiformann.

\item[b)] Regis oppgave er å stå for drift og forvaltning av
Realistforeningens tekniske utstyr.

\item[c)] Regi har ansvaret for å rapportere regnskapsrelevant
  informasjon til Økonomiutvalget og overholde vedtatte budsjetter.
\end{itemize}


\subsection*{§ 18 VAKTGRUPPA}

\begin{itemize}
\item[a)] Vaktgruppa ledes av Vaktgruppesjef som utpekes av Hovedstyret etter
innstilling av Vaktgruppa.

\item[b)] Vaktgruppas oppgave er i samråd med Arrangementstyret og
Kjellerstyret å stå for vakthold under Realistforeningens
arrangementer.

\item[c)] Vaktgruppa har ansvar for Vaktgruppas regnskap og økonomistyring.
\end{itemize}


\subsection*{§ 19 HUSBJØRNREDAKSJONEN}
\begin{itemize}
\item[a)] Husbjørnredaksjonen ledes av Husbjørnredaktør, og kan knytte
  til seg så mange medlemmer som Husbjørnredaktør finner ønskelig.

\item[b)] Husbjørnredaksjonen har ansvar for å utgi avisen
  Husbjørnen, \emph{Ursus Domesticus}.

\item[c)] Husbjørnredaksjonen har ansvar for å rapportere
  regnskapsrelevant informasjon til Økonomiutvalget og overholde
  vedtatte budsjetter. 
\end{itemize}

\subsection*{§ 20 BJØRNEGILDET} 

\begin{itemize}
\item[a)] Bjørnegildet avholdes vårsemesteret hvert tredje år. 

\item[b)] Bjørnegildet ledes av et styre, hvis medlemmer velges av
generalforsamlingen seneste tre semestere før Bjørnegildet.

\item[c)] Bjørnegildestyret består av Gildesjef, Sekretær, Økonomiansvarlig
og så mange medlemmer som generalforsamlingen finner nødvendig.

\item[d)] Bjørnegildestyret har ansvaret for Bjørnegildets regnskap og
økonomistyring.

\item[e)] Formannen i Realistforeningen har møte- og stemmerett i
Bjørnegildestyret.
\end{itemize}


\subsection*{§ 21 KOMITEER}

\begin{itemize}
\item[a)] Komiteer utnevnes og får sitt mandat av en generalforsamling eller
Hovedstyret.

\item[b)] En komité har ansvar for å rapportere regnskapsrelevant informasjon til Økonomiutvalget og overholde vedtatte budsjetter.
\end{itemize}


\subsection*{§ 22 GENERALFORSAMLING}

\begin{itemize}
\item[a)] Generalforsamlingen er foreningens høyeste myndighet i spørsmål som
ikke kommer inn under §15 d) under første punkt. Generalforsamlingen er
beslutningsdyktig når minst 1/10 av medlemmene er tilstede, dog slik
at 50 stemmeberettigede er tilstrekkelig dersom foreningen har mer enn
500 medlemmer.

\item[b)] Ordinær generalforsamling avholdes i andre halvdel av hvert 
semester.  Ekstraordinær generalforsamling avholdes når Hovedstyret 
vedtar det eller det kreves av Desisjonsutvalget eller minst 1/10 av 
medlemmene, dog slik at 50 medlemmer er tilstrekkelig dersom foreningen 
har mer enn 500 medlemmer.

\item[c)] Innkallelse til ordinær og ekstraordinær generalforsamling må være
offentliggjort minst 10 virkedager i forveien.  Ved ordinær og ekstraordinær
generalforsamling må forslag til foreløpig dagsorden være offentliggjort 
senest 5 virkedager i forveien.  Som virkedag regnes alle dager i samme 
semeseter som ikke er helg, offentlig høytidsdag eller feriedag for 
studentene ved Det matematisk- naturvitenskapelige fakultet ved 
Universitetet i Oslo.  Generalforsamlinger innkalles av Hovedstyret.  
Dersom dette ikke fungerer eller ikke etterkommer lovlige krav om at 
generalforsamling skal kalles inn, skal Desisjonsutvalget overta 
Hovedstyrets plikter når det gjelder generalforsamlinger, med unntak av 
§22 e).

\item[d)] Forslag om lovendringer og andre saker som ønskes tatt opp på
generalforsamlingen må være Hovedstyret i hende og offentliggjøres 5
virkedager før.  Desisjonsutvalget kan fremme endringsforslag inntil 48
timer før generalforsamlingen.  Lovendringsforslag kan ikke behandles
på ekstraordinær generalforsamling.

\item[e)] Hovedstyret skal offentliggjøre et sett med budsjettrammer til
Generalforsamlingens overveielse senest 10 virkedager før høstens
Generalforsamling.  Andre forslag til budsjettrammer som ønskes
vedtatt må være Hovedstyret i hende og offentliggjøres senest 48
timer før Generalforsamlingen.

\item[f)] Generalforsamlingen kan foreta endringer i rekkefølgen av punktene
til det endelige forslag til dagsorden.  Den kan også utelukke ett
eller flere av de foreslåtte punkter så lenge det ikke strider mot
§22 l).  Den endelige dagsorden godkjennes av generalforsamlingen.  I
forbindelse med godkjennelse av dagsorden skal det velges ordstyrer,
referent og to medlemmer til å undertegne generalforsamlingens
protokoll.

\item[g)] Ethvert medlem kan på generalforsamlingen foreslå tatt opp saker
utenom den oppsatte dagsorden. Generalforsamlingen kan ikke fatte
vedtak i slike saker.

\item[h)] Generalforsamlingen kan med alminnelig flertall gi ikke-medlemmer
møte- og talerett.

\item[i)] Avstemningen på generalforsamlinger skal være hemmelig når minst
tre stemmeberettigede krever det.  Stemmerett har alle medlemmer av
Realistforeningen som var innmeldt før innkallingen til
generalforsamlingen.

\item[j)] Valgbare til verv i Realistforeningen er alle foreningens
medlemmer, med de innskrenkninger som følger av §14 b), §15 b) og c) og 
§16 b).

\item[k)] Valg på flere tillitsvalgte under ett avgjøres med alminnelig
flertall.  Ved valg på en enkelt tillitsvalgt kan tre
stemmeberettigede kreve at valget skal avgjøres med absolutt
flertall.  Oppnår ingen dette ved første avstemming, avholdes bundet
omvalg.

\item[l)] På ordinær generalforsamling behandles:

        \begin{itemize}
        \item[1.] Regnskaper, etter en redegjørelse for RFs totale økonomi. 
        
        \item[2.] Budsjettrammer.  På høstens generalforsamling vedtas 
                  budsjettrammer for neste år.  På vårens generalforsamling 
                  kan budsjettrammene revideres.

        \item[3.] Eventuelle lovendringsforslag 

        \item[4.] Semesterberetninger 

        \item[5.] Fastsettelse av kontingenter. 

        \item[6.] Valg av tillitsvalgte: 

                \begin{itemize}
                \item[6a)] Formann (§9b) 
    
                \item[6b)] Sekretær (§9b) 

                \item[6c)] Arrangementsjef (§9b) 

                \item[6d)] Fagsjef (§9b) 

                \item[6e)] Kjellermester(§9b) 

                \item[6f)] BlæStsjef (§9b) 

                \item[6g)] Kafesjef (§13a) 

                \item[6h)] To Skjenkemestre (§13a)

                \item[6i)] To medlemmer til Økonomiutvalget (§14a)

                \item[6j)] Leder av Økonomiutvalget (§14a) 

                \item[6k)] Ett medlem til Desisjonsutvalget (§15a) 

                \item[6l)] Minst ett medlem til Revisjonsutvalget (§16a) 

                \item[6m)] Formann i Regi (§17a) 

                \item[6n)] Forretningsforfører i Regi (§17a) 
                \end{itemize}
        \end{itemize}
\end{itemize}


\subsection*{§ 23 MISTILLIT}

Foreningens medlemmer kan fremme mistillitsforslag mot medlemmer av
styrer, utvalg og komiteer unntatt Desisjonsutvalget. Slike forslag
kan bare behandles av en generalforsamling, og må være fremmet 48
timer før generalforsamlingen. Mistillitsforslag vedtas med 2/3
flertall.  Dersom mistillitsforslaget mot et medlem av et organ blir
vedtatt, kan generalforsamlingen vedta å holde nyvalg på samtlige
medlemmer av organet for resten av hvert medlems funksjonstid.


\subsection*{§ 24 LOVENDRINGER}

Forslag til lovendring skal bare behandles på ordinær
generalforsamling, og må få 2/3 flertall blant de tilstedeværende
stemmeberettigede for å vedtas.


\subsection*{§ 25 LOVERS GYLDIGHET}

Disse lovene er gyldige fra den dag de blir vedtatt, slik at alle
tidligere lover opphører å gjelde fra samme dag.



