% vim:fileencoding=utf-8
% -*- encoding: utf-8 -*-
\section{Kaf\'esjefens erfaringer}
%\author{Robert Bauck Hamar}
%\date{Høsten 2002---Våren 2003}

\section{Bestillinger}
Som kaf\'esjef har man ansvar for at kafeen har varer.
\subsection{Tine}
Er vår viktigste leverandør. Fra Tine kjøper vi meieriprodukter.
Det er lurt å ha:
\begin{itemize}
	\item Melk. H-Melk til bafler og kaffe. Kjøpes litersvis, 
		men ti liter om gangen.
	\item Rømme. Lettrømme til bafler. Kjøpes i brett på seks.
	\item Litago. Kjøpes i ti $\times$ halvvlitere.
	\item Iste og juice. Kjøpes i ti $\times$ halvvlitere.
	\item Norvegia. Kjøpes i femkilospakninger; bemerk dette, idet
		pakninger med to femkilososter har samme varenummer.
	\item Smør. Kjøpes i 18 halvkilospakker.
	\item Gudbrandsdalsost.
	\item Kremtopp.
\end{itemize}
Det er lurt å legge merke til at ost og smør ikke står i bestillingslista.

Jeg har pleid å bruke bestillingsskjemaet som ligger i kaf\'esjefhylla på kontoret,
men ringt inn bestilling per telefon. Det første for å holde oversikt over hva som
er bestilt. Det andre fordi Tine må ha bestillinger på minst 500 kroner for å
levere gebyrfritt. Bestillinger må være Tine i hende senest 10.30 dagen før levering.
Leveringer kan man få på tirsdager eller torsdager.

Når det gjelder størrelsen på bestillinger gjelder det å holde seg i nakken. 
Meieriprodukter går gjerne fort ut på dato. Litago holder seg en stund, og
det gjør også uåpnet ost, men andre ting gjør det ikke. Det lønner seg skjelden
å ha mer enn en tipakning juice/iste stående. Litago går heller ikke altfor fort,
med unntak av sjokolade, som det kan gå en hel pakning av på en god dag.
Med det nye tellesystemet, som i skrivende stund ikke er helt innført, bør
det bli enklere å regne ut bestillinger.

Leveringene fra Tine kommer tidlig om morgenen. Som regel blir de dratt inn
av Sandy, en av de ansatte i teknisk avdeling. Ellers må de dras gjennom
gangen fra Abel til VB. Det gjelder å få tingene i kjøleskap fortest mulig.
Etter dette skal tralla og melkebrettene settes i vareheisen i kjelleren i
Abel. For å komme dit tar man heisen ned fra underetasjen, går inn døra
til venstre når man går ut av heisen, går til venstre, rett fram, så første dør på
venstre side.

\subsection{Temperato}
Espressoting får man tak av Temperato. Dette er praktisk talt et enmannsforetak,
men den ene gangen jeg bestilte var han kjapp til å levere. Det er greit å
snakke med mannen, og gjerne bestille i store kvanta. Ting man bør ha:
\begin{itemize}
	\item Espresso. De har litt utvalg. Personlig har jeg foretrukket
		en av de dyrere variantene.
	\item Brunt sukker. Porsjonspakket. Fåes i store kvanta.
	\item Mocchasus(?). Til mocchaer. Ta vare på dispenseren, som
		brukes på kannene. Kan gjerne \`og prøve hvit saus.
	\item Sirup. Se hva som går. Ønske for eggnogsmak er ytret (Ståle).
	\item Glass og kopper.
\end{itemize}
Mannen i Temperato heter Alexander von der Lippe. Han nås på: 
\mbox{\textsf{lippe@temperato.no}}, telefon 66~80~87~03, mobil
91~12~28~99 og faks 66~80~36~24.

\subsection{Andre ting}
Det beste er å kjøpe inn det man trenger i store kvanta. Lån bil hos
budsentralen. Pass på å holde av en bil på forhånd. Stedet vi tidligere
har brukt heter Brubekk Storcash, og ligger på \dots Brubekk. Finn ut 
hvor det er på forhånd, og ta gjerne med kart. Det lønner seg også å
ta med kjellermester, i det minste en annen person. Kundekort er obligatorisk.
På brubekk kan man kjøpe det meste i store kvanta, kanskje litt billigere.

Egg, brød og skinke må nødvendigvis kjøpes andre steder. Gjerne bunnpris. Det er
enklere for kaf\'esjef om kaf\'epersonalet kan gjøre dette, men det kan
være en fordel å gjøre det selv, fordi færre trenger å passe på å huske kvitteringer 
og ha ansvaret for penger. Det blir uansett ting som må kjøpes på
Bunnpris til forskjellige tider, så det er greit at personalet vet hvordan
det funker.

\section{Personalet}
Det er flere kilder til personale. I kafeen har man hatt god tradisjon i å holde på
personalet gjennom flere semestre. Derfor er det gode grunner til å ta vare på
personalet. Få folk til å komme på Mandager$^\text{TM}$, og internarrangementer.
Kranførerkurset tror jeg ikke er veldig viktig å nevne mer enn at det lønner seg
å bli litt kjent med de som er der. En viktig kilde til nye interne har også
vært å verve folk selv. Hvis personen en prøver å verve er litt interessert i
å bli med bør det ikke ta lang tid å overtale. Det er \`og lettere å få med
vennene til folk som allerede jobber i kafeen.

Når det gjelder Baffeluka er det kaf\'esjefens jobb å skaffe folk. Det aller mest
effektive er å spørre folk i kjelleren på mandager og fredager; det er få som
svarer på mail. Bruk et par uker. For din egen del er det enklest å få kabalen
til å gå opp om folk oppgir flere mulige tidspunkt, men folk flest synes visst
det er best å skrive seg opp ett sted. Foran baffeluka våren 2003 brukte jeg
mye tid på å få folk til å skrive seg opp flere steder, men ingen måtte
jobbe mer enn en gang.

Skiftene har tradisjonelt vært 10---12, 12---14 og 14---16. Det er dog ingenting
mot å bruke folk annerledes. Det siste semesteret har det vært stort press på
onsdager mellom 12 og 14; det har da vært lurt å ha tre personer på 12---14-vakta.
Det kan man regne med å måtte gjøre framover også.
En lur ting er å sette opp et skjema tidlig slik at alle vet når man skal jobbe.

\subsection{Organisering}
Det er flere ting å organisere start og slutt. En løsning som har fungert det
siste semesteret er at en av de som står klokka 10 har ansvar for å åpne.
For å overlevere nøkkelen bør denne personen ha tilgang til kontoret.
RF tenker på å kjøpe et låsbart nøkkelskap, og dette kan brukes. Inntil
da får man finne på noe annet lurt.

Når det gjelder å stenge har jeg valgt å gjøre det selv. En grunn til det
er at færrest mulig skal trenge å kunne gjøre feil i regnskapet. Men
det går selvfølgelig an å delegere ansvar videre. Jeg vil bare peke
på at skjenkemestrene også tar oppgjørene sine selv.

\section{Diverse}
\subsection{Baffeluka}
For å få Baffeluka til å bli ei god Baffeluke, er det noen ting
som må på plass. Jeg har allerede nevnt personale. Videre er det viktig
at folk vet det er Baffeluke. Blæst kan stå for blæsting. Send gjerne en 
mail til blæst eller blæstsjef. 

Det aller viktigste er at Regi vet når
Baffeluka er. Jeg snakker av erfaring: Send mail, ikke bare snakk med folk.
Gjerne også påminnelser. Som Sturle sa: «Regi er treige,» de følger ikke like
godt med som andre grupperinger. 

Nummer tre: filmer. Lene Kittelsen er gjerne en person som er kyndig til å
plukke ut filmer. Finn filmer i god tid, slik at det er lettere å få tak i dem.
De interne kan gjerne også bli hørt. Det holder med to filmer per dag, eventuelt
\'en dersom den er lang. En god baffelukefilm er en film man kan se på uten å se
hele. Den siste filmen er alltid «Kellys helter.» Regi har den på VHS.

Baffeluka grunner i at kjelleren var åpen
fordi folk ville se OL. OL og Ski-VM pleier gjerne å gå sammen
med Baffeluka på vårsemesteret. På høstsemesteret kan man gjerne også vise
TV med et idrettsarrangement.

\subsection{Ting som bør gjøres}
Flere ting bør gjøres. Først og fremst bør man lage en stillingsinstruks for
personalet. Den bør for eksempel inneholde instruksjoner om hvordan maskiner
fungerer og andre ting. Gjerne også Baffelrøreoppskriften. Etter et 
kjellerstyrevedtak er oppskriften hemmelig. Som en morsom avledningsmanøver,
bør man kanskje fake oppskriften på en måte alle personale kjenner til.
For eksempel kan ølet byttes ut med Bache Gabrielsen VSOP.

Man bør også holde orden i skapene. En id\'e er å lage oversikt over hva som 
befinner seg i skapene. Ting som andre kan ha glede av, for eksempel vannkoker
og kaffe, bør ikke låses inn. Apropos kaffe: Jeg har innført en ordning der
folk som tar kaffe putter penger på et glass. Glasset kan tømmes av og til,
og pengene brukes til ny kaffe.\footnote{Denne kaffen er allerede inne i regnskapet,
så man trenger da ikke kvittering/bilag.} Pass på at folk ser glasset, og forstår
ordningen.

Når det gjelder kjøleskapene, har man lite plass. Hvis det trengs, kan man ta
ting som ikke trenger å holdes kaldt ut. For telling og praktiske formål, er
det lurt å holde orden.

Få tak i noe til oppvaskmaskina, som kan håndtere bestikk. Jeg har sett hele 
rister med finmasket nett. Til vårt forbruk holder det nok med en bestikkurv
fra en vanlig oppvaskmaskin.

\subsection{Praktisk info}
Oppvaskmaskinen er nå koblet opp mot såpetanken, slik at såpen egenhendig
blir sugd inn i maskina. Den må nok kalibreres, og det står en guide til
dette i manualen. Påse at det er nok såpe i dunken. Såpe kjøpes også
på Brobekk. Ølvaskeren bruker fortsatt krystallsoda.
