% vim:fileencoding=utf-8
% -*- encoding: utf-8 -*-
\section{Kaf\'e}
%\author{Kine V. Lund}
%\date{Våren 2011}

\subsection{Bestillinger}
Som kaf\'esjef har man ansvar for at kafeen har varer.
\subsubsection{Tine}
Er vår viktigste leverandør. Fra Tine kjøper vi meieriprodukter.
Det er lurt å ha:
\begin{itemize}
    \item Melk. H-Melk til bafler og kaffe.
    \item Rømme. Lettrømme til bafler.
    \item Smør.
    \item Gudbrandsdalsost.
    \item Kremtopp.
    \item Laktosefri melk.
    \item Go'morgen-yoghurt
    \item Lett-melk
\end{itemize}

Bestillinger gjøres i nettskjema, tidligere kafésjef har ansvar for å videreføre
brukernavn og passord. Dersom det uventet ikke fungerer å logge seg inn, kan det
hende det er fordi vi f.eks. ikke har bestilt noe på en stund (det har vært
sommerferie e.l.). Da er det bare å ringe leverandøren og forklare saken, så
burde brukeren bli aktivert igjen.

Når det gjelder størrelsen på bestillinger gjelder det å holde seg i nakken.
Meieriprodukter går gjerne fort ut på dato. Leveringene fra Tine kommer tidlig
om morgenen. Som regel blir de dratt inn av en av de ansatte i teknisk avdeling.
Ellers må de dras gjennom gangen fra Abel til VB. Det gjelder å få tingene i
kjøleskap fortest mulig. Etter dette skal tralla og melkebrettene settes utenfor
garasjedøra ved søppelkassene bak Abel.

\subsubsection{Temperato}
Espressoting får man tak av Temperato, og dette ansvaret blir som regel delegert
til en kaffe- og te-ansvarlig. Ting man bør ha:
\begin{itemize}
    \item Espresso.
    \item Sirup. Selger ikke spesielt bra, så se hva som går og ikke ha for mye.
    \item Glass og kopper, men dette kan kjøpes billigere på f.eks. Ikea.
\end{itemize}

\subsubsection{ASKO}
ASKO selger det meste, og ting du bør ha derfra er:
\begin{itemize}
    \item Egg
    \item Hvetemel
    \item Sukker
    \item Honning, suketter o.l.
    \item Wienertoast
    \item Sjokoladepellets
\end{itemize}


\subsubsection{Andre ting} Vær forsiktig med å handle på Bunnpris, for det kan
bli dyrt i lengden. Husk å få de dagansvrlige til å legge tilbake Bunnpriskortet
etter hver gang de har brukt det, og alltid føre bilag med engang. Da blir ØU
glade. Det er også viktig at kafésjef holder orden på hvor mye jobbgoder som er
blitt benyttet. Da blir også ØU glade.

\subsection{Personalet} Det er flere kilder til personale. I kafeen har man hatt
god tradisjon i å holde på personalet gjennom flere semestre. Derfor er det gode
grunner til å ta vare på personalet. Få folk til å komme på
Mandager$^\text{TM}$, og internarrangementer.  Kranførerkurset tror jeg ikke er
veldig viktig å nevne mer enn at det lønner seg å bli litt kjent med de som er
der. En viktig kilde til nye interne har også vært å verve folk selv. Hvis
personen en prøver å verve er litt interessert i å bli med bør det ikke ta lang
tid å overtale. Det er \`og lettere å få med vennene til folk som allerede
jobber i kafeen. De som allerede har meldt seg vil nok også ha lettere for å
faktisk komme på jobb når de skal (større ansvarsfølelse heter det visstnok)
dersom du som sjef ringer alle de som har meldt seg som funker i begynnelsen av
semesteret. Dette gjelder spesielt på høstsemesteret.

Når det gjelder Baffeluka er det kaf\'esjefens jobb å skaffe folk. Det aller
mest effektive er å spørre folk i kjelleren på mandager og fredager; det er få
som svarer på mail, men send gjerne en mail på interne-lista også. Bruk et par
uker. For din egen del er det enklest å få kabalen til å gå opp om folk oppgir
flere mulige tidspunkt, men folk flest synes visst det er best å skrive seg opp
ett sted.

Skiftene har tradisjonelt vært 10---12, 12---14 og 14---16. Det er dog ingenting
mot å bruke folk annerledes. På tidligskiftet pleier det å holde med to funker,
men det er fint om de kan komme minst et kvarter før for å hjelpe dagansvarlig
med å ha alt klart til åpning.  Midtskiftet bør ha tre funker, og seinskiftet
bør ha tre funker pluss dagansvarlig. Funkene bør være inneforstått med at det
forventes at de blir igjen i en halvtimes tid for å hjelpe til med
stengerutinene.

\subsubsection{Dagansvarlig} Ettersom kafeen er blitt så stor som den har blitt,
lønner det seg å ha noen dagansvarlige og en kaffe- og te-ansvarlig. Det er
praktisk at hver dagansvarlig har ansvar for hver sin faste dag i uka. Så fire
dager åpent i uke vil tilsi fire dagansvarlige. De dagansvarlige, kaffe- og
te-ansvarlig og kafésjef utgjør til sammen kafégruppa, populært kalt
''kaféstyret''.

Dagansvarlig har ansvar for å møte tidlig nok, sørge for at det er baffelrøre
klart til åpningstid, gjerne med et par baffelplater klare allerede, men det
aller viktigste er selvfølgelig at kaffen er klar for trøtte studenter tidlig på
morran. Det er helt ok å få funkene til å ordne dette, men ansvaret faller på
dagansvarlig.

Dagansvarlig har koden til safen, så hun skal låse ut kassa og helst være
tilgjengelig in persona eller på telefon hele dagen i tilfelle det f.eks. skulle
mangle veksel, eller om det er mangel på varer.

Det er vanlig at dagansvarlig fungerer som vanlig funk siste skiftet, men at hun
tar initiativ i forhold til rydding og vasking mot slutten. Husk å delegere bort
ansvar, for det er ikke alltid funkene føler seg sikre nok til å se hva som
trengs å gjøres av seg selv. Det er vanlig at dagansvarlig tar oppgjøret mens
funkene vasker gulvet, støvsuger osv. etter kafeens stengetid.

\subsection{Diverse} \subsubsection{Baffeluka} For å få Baffeluka til å bli ei
god Baffeluke, er det noen ting som må på plass. Jeg har allerede nevnt
personale. Videre er det viktig at folk vet det er Baffeluke. Blæst kan stå for
blæsting. Send gjerne en mail til blæst eller blæstsjef, gjerne et par uker før.
Da får de god tid til å lage plakater.

Det aller viktigste er at Regi vet når Baffeluka er. Jeg snakker av erfaring:
Send mail, ikke bare snakk med folk.  Gjerne også påminnelser. Som Sturle sa:
«Regi er treige,» de følger ikke like godt med som andre grupperinger. Men, så
fort de har mottatt beskjeden er de svært behjelpelige når det kommer til å
sette opp ekstra høyttalere og de sørger for at projektoren fungerer. (Det er
forresten viktig at Regi får beskjed om at sistnevnte skal brukes selv hvis du
ikke trenger deres ekspertise. De liker å vite når resten av foreningen bruker
tingene deres.)

Nummer tre: filmer. Det er vanlig å vise én film per skift. Altså tre filmer
hver dag. Det er også vanlig å holde opp alle fem dagene i Baffeluka, og den
siste filmen på fredagen SKAL være ''Kelly's Heroes''. Hvis ikke er det fare for
straff for kafésjef.

Baffeluka grunner i at kjelleren var åpen fordi folk ville se OL. OL og Ski-VM
går gjerne sammen med Baffeluka på vårsemesteret. På høstsemesteret kan man
gjerne også vise TV med et idrettsarrangement. Men i nyere tid har det vært
vanligere å ha Baffeluka som den første eller andre uka kafeen er oppe det
inneværende semesteret.

\subsubsection{Ting som bør gjøres} Flere ting bør gjøres. Først og fremst bør
man lage en stillingsinstruks for personalet. Den bør for eksempel inneholde
instruksjoner om hvordan maskiner fungerer og andre ting. Gjerne også
Baffelrøreoppskriften. Etter et kjellerstyrevedtak er oppskriften hemmelig. Som
en morsom avledningsmanøver, bør man kanskje fake oppskriften på en måte alle
personale kjenner til.  For eksempel kan ølet byttes ut med Bache Gabrielsen
VSOP.

Man bør også holde orden i skapene. En id\'e er å lage oversikt over hva som
befinner seg i skapene. Ting som andre kan ha glede av, for eksempel vannkoker
og kaffe, bør ikke låses inn.

Sørg for at Kjellermester sender kluter og mopper til vasking med jevne
mellomrom. Siden vaskefolka er så greie at de gidder å vaske dem for oss, bør vi
være flinke til å ikke drøye det til klutene mugner i søppelsekken før vi sender
de til dem.
