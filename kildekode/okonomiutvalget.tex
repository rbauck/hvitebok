% vim:fileencoding=utf-8
% -*- encoding: utf-8 -*-
\section{Økonomiutvalget}

\subsection{ØU-leder}

\subsubsection{Stillingsbeskrivelse}

\begin{itemize}
\item{} leder økonomiutvalget og derfor overordnet ansvar for
  utvalget. Dette vil si å kalle inn til møter, finne møtedager som
  passer for de fleste.
\item{} skal ha oversikt over RFs totale økonomi. Dette betyr at
  vedkommende skal kunne gjøre rede for RFs økonomiske situasjon enn
  når dette måtte forekomme.
\item{} skal sørge for at de forskjellige utvalgene fører en
  ansvarsfull økonomi. Dette gjøres gjerne ved å lære utvalgene opp
  etter RFs gjeldende regler, rutiner og system dersom det er behov
  for dette.
\item{} informere de andre utvalgene hvordan de ligger ann ifølge
  budsjett. Dette er viktig for at man til enhver tid holder seg
  akkurat eller i nærheten av budsjettet.
\item{} holder bilagskurs for de som gjennom sitt verv trenger å
  skrive bilag. Dette er en meget viktig arbeidsoppgave. Grunnen til
  dette er at man ønsker en så homogen som mulig bilagsskriving slik
  at regnskapet blir oversiktlig å lese for eventuelle utenforstående.
\item{} delegerer oppgaver til medlemmene. Som en god leder gjelder
  denne oppgaven som alltid. Det er viktig å prøve å motivere
  medlemmer til å synes det er greit å være med i ØU.
\item{} passer på at regninger blir betalt. Alle regninger skal
  betaler før de går ut på dato, rimelig nok.
\item{} sørger for at MVA blir betalt. Dette er et veldig viktig
  punkt. MVA-oppgaven/omsetningsoppgaven er RF sitt ansikt utad mot
  skattemyndighetene. Vi får igjen skatt på de varene iv kjøper inn som er
  momsbelagt og vi betaler moms for de varene vi selger igjen. I
  RF-sammenheng gjelder dette mye leskende drikker.
\item{} snakke med bank og andre instanser i spørsmål som har med
  økonomi å gjøre. Siden de som sitter i ØU sjelden har så mye
  erfaring med økonomi, er det viktig å være etterrettelig å prøve å
  få ut informasjon der det er mulig. Mest av alt gjelder det å spørre
  tidligere ØU-medlemmer (derav ledere og sekretærer). 
\item{} skrive ut bilagsforsider. Dette er en ren
  rutinejobb. Bilagsforsidene trengs for de fleste typer bilag.
\end{itemize}

\subsubsection{Rutinebeskrivelse}

\begin{itemize}

\item{} Etter at det nye hovedstyret er satt, har ØU-leder ansvar for å kalle inn de
  relevante og gå i banken for å fornye tilgang til kontoene. Dette
  kan bare gjøres med et underskrevet generalforsamlingsreferat i
  hånden. ØU-leder kaller sammen de i hovedstyret det gjelder, for
  øyeblikket Kjellermester, Arrmester og Formann. ØU-leder skal ha
  tilgang til alle kontoer. Kjellermester skal ha tilgang til
  kjellerkonto, Arrmester arrkonto og Formann sparekonto. Sparekontoen
  skal bare kunne manipuleres med både Formann og ØU-leder tilstede.

\item{} \textbf{Betaling av fakturaer} Dette skjer ved hjelp av
  postgiro og signeres på vanlig måte når dette er ordnet i banken.

%\end{itemize}
\item{} Innlevering av MVA-/omsetningsoppgave. Skattemyndighetene skal ha kontroll
  over all omsetning RF har. Derfor leverer vi MVA-oppgave hver annen
  måned. Alle ark som trengs blir sendt fra Flykesskattesjefen og
  innleveringsfristen er som regel den 10. i hver partallsmåned. I
  forbindelse med innleveringen er det innført interne
  kontrollrutiner. Disse er å finne på et ark i en av
  permene. De viktigste rutinene er
  \begin{itemize}
  \item{}telling av alle kasser. For å finne ut om summen i kassen
    stemmer med regnskapet.
  \item{}bank avstemning. Se at det vi har ført i bankpostene faktisk
    stemmer med bevegelse i banken.
  \item{}leverandør- og kundeavstemning. Dette betyr at man må gå
    gjennom alle kunder og leverandører (regnskapet) og sjekke om de går i null
    eller ikke. Dette er også noe som skal gjøres etter hver
    postering.
  \item{}sjekk om penger fra kassa er ført inn i regnskapet.
  \end{itemize}
\end{itemize}
Denne jobben tar faktisk litt tid . Det er derfor viktig
å begynne med disse sjekkrutinene i god tid før
MVA-oppgaveinnlevering.
\begin{itemize}
\item{}\textbf{Avslutting av perioder} Hver periode (måned) skal
  avsluttes i regnskapsprogrammet. Dette skjer som regel i forbindelse
  med momsinnbetalingen. 

\item{} Avslutting av år. Rutinene er de samme som for avslutning av
  perioder. Den eneste store forskjellen er at man skal sjekke om hele
  året stemmer. Eventuelle differanser, som av forskjellige grunner
  ikke kan føres bort, bør beskrives i et vedlegg til
  regnskapet. Permer med samtlige bilag settes opp i bokhylle for
  eventuelt ettersyn.

\item{} Putte bilag i perm. Alle bilag puttes i en av 3 permer før
  føring. Hver av permene har sin egen nummerserie som identifiserer
  typen bilag ført.

  \begin{itemize}
  \item{} \textbf{hovedboksbilag} Alle bilag som har med bevegelse i
    bank å gjøre. Dette gjelder f. eks. nattsafeposetellinger levert fra
    bank via fax og kvitteringer for betaling av giroer osv.
    
  \item{} \textbf{kontantsalgbilag} Alle bilag som har med salg i
    baren/kafeen å gjøre. Disse bilagene inneholder hvor mye penger som
    ble talt opp kontra hva kassen mente. Differansen blir
    svinn. Pengene som gikk inn i kassen føres mot riktige salg-poster i
    regnskapet. Kvittering for nattsafepose skal stiftes på bilaget,
    samt Z-rapporter fra kassa. Summen som var opptalt i kassa skal være
    lik summen på nattsafeposen. Dersom dette ikke er tilfelle skal den
    som skrev bilaget kontaktes for å gi en forklaring på forskjellen.
    
  \item{} \textbf{inngående-faktura-bilag} Andre bilag;
    betalingsmeldinger, utlegg, osv. Disse føres mot leverandører og
    kunder. 
    \begin{quote}
      \textbf{Når en leverandør har levert varer til Kjellern skal kjellermester betale gjeldene faktura.
        Kopien (merket kopi eller kopiert av RF) skal gis
        ØU-leder. Dette for å unngå dobbeltføringer.} 
    \end{quote}
  \end{itemize}
\item{}Ved semesterstart skal ØU-leder skifte alle koder på alle safer
  for å hindre at de som ikke skal ha tilgang til safer har det.  
\end{itemize}

\subsection{ØU-sekretær}

\subsubsection{Stillingsbeskrivelse}

\begin{itemize}
\item{} Skriver referat fra ØU-møter hvis behov. Som regel er det ikke
  behov for referater. De fleste ØU-møter går ut på å føre bilag.
\item{} Hjelper til med føring av bilag og hjelper
  til med å lære opp nye medlemmer. Det å styre et regnskap tar litt
  tid å sette seg inn i. Derfor er det veldig viktig at nye medlemmer
  får god opplæring.
\item{} Sørger for og administrerer full lagertelling. For å ha
  kontroll over svinn i lageret er det viktig med månedlig
  lagertelling. Å administrere telling innebærer å avtale et passende
  tidspunkt med kjellermester samt utføre tellingen. Resultatet av
  tellingen føres på eget ark (finnes i exel-format), hvor all
  summering gjøres automatisk. Tellingen føres også inn i regnskapet
  for å kunne sammenligne poster for innkjøp og salg av diverse
  varer. Lagertellingsresultatet skal typisk være differansen mellom
  innkjøp og salg.
\end{itemize}

\subsection{Rutinebeskrivelse}
\begin{itemize}
\item{} møte opp på møter eller melde forfall.
\end{itemize}


\subsection{ØU-medlem}
\subsubsection{Stillingsbeskrivelse}

\begin{itemize}
\item{} lærer å føre bilag og å forstå hvordan rutinene er
\item{} hjelper til 
\end{itemize}

\subsubsection{Rutinebeskrivelse}
\begin{itemize}
\item{} møte opp på møter eller melde forfall
\end{itemize}

\subsection{Erfaringsprotokoll}
Erfaringsmessig er det vanskelig å følge alle rutiner til punkt og
prikke. Siden dette er en studentforening hvor vervene faktisk gjøres
på fritiden og utenfor skole, kan man ikke forvente at alt skal
gjøres. Stillings- og ruteniebeskrivelsen er ment som veiledende. Det
viktigste er at ØU har kontroll over alle bevegelser i regnskapet og
en oversikt over foreningens økonomi, samt leverer inn omsetningsoppgave. 
