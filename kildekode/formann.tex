% vim:fileencoding=utf-8
% -*- encoding: utf-8 -*-
\section{Formann}

\subsection{Mandat}
Formannen leder Hovedstyret. Formannen er foreningens ansikt utad og skal sørge for at
kontakten med andre studentforeninger, SiO Læringsmiljø og Universitetet opprettholdes.

Formannen har ansvar for å oppdatere denne teksten med nye erfaringer etter fullført verv.
Formannen har også et overordnet ansvar for at alle tekstene i hviteboken blir oppdatert
etter erfaringer og rutineendringer gjennom og spesielt mot slutten av semesteret.


\subsection{Oppgaver}

\subsubsection{Hastesaker ved semesterstart}
\begin{itemize}
\item Brønnøysund. Kan først fikses etter at genforsreferatet er klart.
\item Tilgang til bygg. Liste med navn og kortnummer sendes
      til teknisk-nordre@admin.uio.no. Husk å dele listen i
      to deler: de som trenger tilgang til VB og Abel, og de
      som kun trenger tilgang til VB.
\item Internliste og internkort. Så fort listene med nye funksjonærer er innhentet,
	oppdater internlisten og internkort skrives ut. Bruk en annen farge på internkortene
	enn de tre siste semesterene. Internkort bør være ute senest på semesterstartfesten.
\item Sørg for å få møtevirksomhet i gang, og sørg spesielt for at AS, KS og IS er i rute m.h.p.
	de første arrangementene.
\item Sørg for å bli meldt på de forskjellige maillistene (typisk: kafe, kjellerstyret, as, okonomi, styret).
\item Bytt lås til kontoret. Hvis det ikke ligger flere sett med nøkkelkort i skapet, så bestill nye sett fra TrioVing.
\item Kranførerkurset. Sørg for at dette blir planlagt godt. Husk spesielt på at dette er en introduksjon til foreningen
	og foreningslivet på matnat, og prøv i minst mulig grad å gi inntrykk av at vi skal "kapre" så mange av de nye
	som mulig til foreningsarbeid. Fint om omvisning og øltapping foregår under ordende former, og ikke totalt
	kaos som det pleier å være. Fortell også litt om andre grupperinger av foreningen enn kjellerstyret,
	men ikke for mye om de delene det er uaktuelt å komme rett inn i. Bar, kafe og blæst er hovedmålet,
	mens regi, vaktgruppa og arr godt kan nevnes kort.

	Hold introduksjonstalen kort, og fokuser på fordelene med å være med i en studentforening
	(bredt sosialt miljø, diverse internarrangementer osv) fremfor arbeidsoppgavene.
	Men det er viktig å få frem hvor mye (lite!) jobb det er å være funksjonær.
	Aktiviser alle sammen så mye som mulig. Sørg for at alle som skal presenteres er klar
	over programmet på forhånd.

    Ha internansvarlig med på planleggingen, og sørg for at denne blir tydelig presentert.
    Sørg for å ikke miste listene etter kurset! Sett gjerne en person til å ta
    ansvar for listene og svare på spørsmål når folk kommer for å skrive seg på (feks internansvarlig).
\end{itemize}


\subsubsection{Generelt}
\begin{itemize}
\item Sørg for at Arrstyret, Kjellerstyret og Infostyret holder frekvente møter og at relevante saker blir tatt opp.
	Pass også på at Fagstyret produserer noe, da spesielt at bedriftsansvarlig opprettholder våre kontakter.
	Det aller viktigste er at kommunikasjonen mellom alle disse ledd fungerer, særlig mellom as/fagstyret
	og infostyret. Sørg spesielt for at infostyret starter i god tid med blæsting av store arrangementer,
	slik at det også er tid til litt forsinkelser (som det ofte blir).
\item I hovedstyret bør diskuteres langsiktige saker, eller prekære internsaker. Hold saker som egentlig bør
	omhandles i andre styrer utenfor, og hold orienteringer til et minimum (hvis det ikke eksplisitt berører
	hovedstyremedlemmene).
\item Hold rede på hvilke kontakter foreningen har på fakultetet, teknisk avdeling, eksternt og med andre foreninger.
	Lag gjerne en liste med disse navnene og hvilken tilknytning de har til foreningen.
	Husk å oppdatere denne listen hvert semester. Få hele hovedstyret til å bruke korrespondansepermen,
	slik at vi får samlet viktige papirer og brev på ett sted.

	(Å skaffe samlet liste over foreningskontakter gjennom Cosa Nostra kan ta tid, og vervene skifter fort,
	det kan derfor lønne seg å spørre ledere direkte ansikt til ansikt om kontaktinformasjon).
\item Sørg for at posthyllene blir ryddet ved begynnelsen av hvert semester. Fjern hyller som ikke lenger er i bruk.
\item Innkalling til Hovedstyremøter skal sendes ut minst tre dager i forveien (se lovene).
	Send gjerne først et forslag til dagsorden fem dager før, slik at andre i HS kan foreslå saker de ønsker
	å ta opp. Send deretter innkalling med dagsorden til internelista.
\item Kontorvakter. Sørg for at det på alle hverdager er en kontorvakt i tidsrommet 12-13 (kan tøyes om nødvendig).
	Denne skal hente post (fra postterminalen) og internpost (fra ekspedisjonen på Abel), sortere og notere all
	post i postloggen. Vedkommende skal også rydde kontoret ved nødvendighet (og evt vaske), samt ta imot
	telefonbeskjeder og gi disse beskjedene videre til korrekt mottaker via mail.
\item Pass på at internarrangementer ikke blir glemt, sett helst datoer sammen med resten av programmet.
\item Stille på møter i Cosa Nostra. Kommuniser viktige saker som diskuteres der til Hovedstyret.
\item Sørge for at semesterberetninger er levert inn senest en uke før generalforsamling.
\item Innkalle til Generalforsamling senest 14 dager før generalforsamlingen (se lovene).
	Husk å bestille rom, og kopiere opp dagsorden, semesterberetninger, regnskap og budsjett til alle.
\item Etter generalforsamlingen: Hold på de avtroppende og sørg for at påtroppende i alle verv får nok opplæring.
	Ha klart programutkast for neste semester før eksamensperioden begynner. Etter eksamen kan folk fort forsvinne.
	Spesielt viktig er det å reservere hytte (minst?) et halvt år i forveien. Ved slutten av høstsemesteret, la
	nye og gamle sammen lage budsjett for neste år i samarbeid med ØU. Ved slutten av vårsemesteret, sørg for
	at alle i HS er forberedte på semesterstarten, da den krever sitt.
\end{itemize}


\subsubsection{Andre oppgaver}
I tillegg til de vanlige oppgavene, som noen ganger blir mye og noen ganger lite, har du som Formann mulighet
til å påvirke fokuset i foreningen mot saker du synes er viktig. Noen forslag til oppgaver det neste semesteret:
\begin{itemize}
\item Oppdatere og forbedre hviteboken (skal gjøres hvert semester!)
\item Lage rutiner for bedre opplæring på alle nivåer. Intern "fadderordning"?
\item Brannsikkerhet, rutiner ved brann. Dette er en viktig sak som vi må bli bedre på.
\end{itemize}


\subsection{Tips og triks}
