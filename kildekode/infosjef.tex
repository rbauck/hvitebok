% vim:fileencoding=utf-8
% -*- encoding: utf-8 -*-
\section{Stillingsbeskrivelse for infosjef og blæstgruppa}

Infosjefs hovedoppgave er å informere eksterne og interne om hva som skjer i kjellern.
Dette er en viktig oppgave, som ikke kan gjøres av en person alene.
Derfor har infosjef en webansvarlig, en listeansvarlig og en blæstgruppe med seg på laget.
Blæstgruppa burde være på minst ca 5 personer, jo flere jo bedre. Medlemmer bør helst rekruteres på slutten av forrige semester, men semesterstart kan også være en fin tid for å utvide gruppa. Spesielt gjelder dette høstsemesteret, når det er mange nye studenter. Det kan være greit å bruke en del tid på å snakke med nye studenter på de forskjellige arrangementene, og informere om RF generelt, og blæst spesielt.

Infosjef kaller inn til blæstmøter når det trengs, og helst i god tid før møtet
(min. 2 arbeidsdager før). Møtene brukes som oftest til å fordele oppgaver,
eller diskutere forskjellige virkemidler man vil bruke i tiden fremover.
Blæsting bør være oppe senest en uke før arrangementet, så det gjelder å planlegge
tidsbruken god tid i forveien.


Vanlige PR-metoder:
\begin{itemize}
\item Banner på Frederikke
\item Plakater over hele Blindern
\item Flyers
\item Stuntblæsting
\item RF's mailingliste
\item Nettsiden vår
\item Universitas
\item UiOs nettsider
\end{itemize}


\subsection{Bannere}
Husk å kapre plass så fort som mulig tidlig i semesteret. Det er to muligheter for å lage bannere.
\subsubsection{Male selv}
(Dette er den tradisjonelle metoden for å lage bannere)
Når man skal male banner kan det være lurt å tegne opp med kritt først. Da 
skriver man ut et lysark (som ligger i blæstskapet) med tekst og bilde på.
For så å henge opp bannerstoffet og rette overheaden mot det, så er det bare
å tegne av. Deretter legger man først ut plast på dansegulvet, for så å male
banneret oppå plasten. Plasten ligger typisk i blæstskapet, eller i en av 
benkene i kjeller'n.
\subsubsection{Bestille}
Fler og fler begynner å bestille trykte bannere på plast. Dette ser proffere ut, og kan være mer iøynefallende. Det er dog noe dyrere, og bør ikke brukes for ofte. Kan f.eks. brukes for hovedbanneret som skal henge der hele semesteret.
\subsubsection{Henge opp banner}
For å henge opp banner på Frederikke kan det være kjekt med en lang 
gardintrapp. Det kan man enten få lånt i resepsjonen på idret, eller i
informasjonsdisken på Abel. Stiftemaskin ligger i blæstskapet på VB.


\subsection{Plakater}
For å få til fine plakater er det lurt å få tak i noen som enten
er flinke til å tegne for hånd eller er dyktige med datagrafikk.
Gjerne begge deler. Det kan kanskje være en fordel å ha et gjennomgående
tema i plakatenes utseende, men ikke lag dem for like. Da blir
de fort oversett etter hvert.

For å få skrevet ut plakater kan printeren ''Goliat'' benyttes. Den står på 
ifi, og man må fylle ut et bestillingsskjema for å bruke den. 
Bestillingsskjema kan man finne på:
\url{http://www.usit.uio.no/it/utskrift/bestilling\_goliat.html}
annen nyttig informasjon om goliat kan finnes på:
\url{http://www.usit.uio.no/it/utskrift/goliat.html}
eller ring operatørene på (228)52501.
Det finnes også printere for å skrive ut i farger, og for å skrive ut på begge sider på ifi.

Blæstfunksjonærene bør tildeles ruter på Blindern der de har ansvaret
for at det henges opp plakater. Infosjef kan f.eks legge bunker med
plakater på kontoret eller i glassmonteren utenfor kjellern, og så
kan funksjonærene henge dem opp når hver og en har tid.

Husk å ta ned plakater etter at arrangementet er ferdig. Dette er
veldig viktig, men blir sjelden gjort. Det ser uprofesjonelt ut når det
henger oppe gamle plakater, samt at folk kanskje ikke gidder å se på dem
fordi de som regel er gamle.

Plakater bør henges opp i god tid før arrangementet! 1-2 uker kan være passe, men hvis de henger for lenge blir de klistret over innen arrangementet begynner. Isåfall kan man gå en liten ekstratur noen dager før og henge opp ekstra plakater.

\subsection{Flyers}
Når man skal lage flyers er det lurt å bruke kuttemaskin for å dele opp
arkene man har skrevet dem ut på. Det finnes en kuttemaskin på ifi,
og det skal også skal finnes en hos SiO Læringsmiljø.

I forkant av store arrangementer går det an å legge ut løpesedler på
lesesaler. Man må da spørre lesesalsvaktene (evt. RSU hvis man vil legge
ut i VB) og man må komme når de åpner. Man må også samle dem inn igjen når
lesesalen stenger. Regner med at det samme gjelder for termstuer.

Andre steder der flyers kan deles ut er på spisestedene på Frederikke, og i andre fellesarealer der folk oppholder seg (slik som i aulaen på VB).

\subsection{Web og mail}
Nettsidene våre er avhengige av oppdatert informasjon for at de skal være
interessante. Hvis det hele tiden ligger aktuell informasjon der, kommer
folk igjen senere. Lag derfor små tekster til alle arrangementer, og
send dem til webansvarlig sammen med, eller i etterkant av, programmet
for semesteret. Nettsidene våre viser de neste to-tre ukene med program
automatisk, så hele semesterets program burde legges inn med en gang det
er klart.

Send ut informasjon på informasjon@rf.uio.no. Denne lista bør brukes i
forkant av alle slags arrangementer. En jevnlig nyhetsmail (f eks ukentlig)
er en god ide. Det er også viktig å være aktiv med å få interesserte til å melde seg på.
Lister på oppslagstavler (husk å hente dem inn!), flyers med nettadressen vår,
lister i døra ved arrangementer, lister i kafeen, mulighetene er mange for
å finne interesserte.

Legg ut informasjon på UiO's "oppslagstavle". http://www.uio.no/tavle/
Her må man få tilgang til å legge ut informasjon, og det fås ved å sende
mail til de som drifter tavla. Info finnes på websiden.


\subsection{Eksterne informasjonskilder}
Få med alle arrangementer (kafé, foredrag, kveldsarr) i "På plakaten" i
Universitas, det er gratis, og det mest selvfølgelige stedet av dem alle
å være synlig. Straks programmet for semesteret er ferdig, kan det være lurt
å sende inn hele lista med korte tekster (liten plass) til hvert arrangement,
slik at de kan legge det inn i systemet sitt.

Ha gjerne en spalte i Universitas i forbindelse med store arrangementer.
Dette koster penger (ca 800,-). Informasjon om dette står i Universitas.

Ved store arrangementer hvor foajeen er i bruk kan det være aktuelt å
gå utenfor Blindern. Men vær forsiktig med dette, da for det første
studenter er (og skal være) målgruppen vår, og dette kan tiltrekke seg
uønskede elementer. Enkelte foreninger har prøvd dette, og droppet det
etter dårlige erfaringer. Send pressemeldinger til andre aviser/radio/TV etc.
Aktuelle steder kan være spalten "Hva skjer?" i Aften Aften, og Dagbladet
og VG har vel tilsvarende. Lag gjerne en liste over stder å sende til,
som kan leveres videre til neste infosjef.


\subsection{Stuntblæsting}
Med stuntblæsting menes som oftest å gå på forelesninger og snakke litt i pausen.
Foreleserne er som oftest positivt innstilt om man spør på forhånd og ikke bruker
for mye tid. Stuntblæsting kan også betegne andre aktive opplegg, det er bare å
være kreativ. Biørneblæs er ofte interesserte i å være med og stuntblæste.


\subsection{Andre oppgaver}
Infostyret har også enkelte andre oppgaver i RF, spesielt å produsere
medlemskort og billetter til arrangementer. Medlemskortene må være klare
før semesterstart, slik at man får forhåndssolgt ved semsterstartsfesten
og ved rekrutteringsstands tidlig i semesteret (f eks diverse fadderopplegg).

Billettene må være klare tidlig hvis man har tenkt å ha forhåndssalg
(som ofte kan være en god ide ved store arrangementer). Billettene må
nummereres slik at vi kan se hvor mange som er solgt. Anders Schau Knatten har et skript som genererer numererte billetter klare til kutting.


\subsection{Diverse tips}
Husk at kafeen er et fint sted å blæste. Her kan man dele ut løpesedler
etc. Avtal gjerne med kafesjef at den (eller kafefunksjonærene) legger ut
løpesedler på bordene når kafeen åpnes. Dette kan også gjøres på
kjellerkroer, men det har en tendens til å skape endel rot. Folk tenner
på løpesedlene, lager papirfly av dem, heller stearin på dem etc.

Infosjef bør ta vare på alt blæstmateriell som lages (i elektronisk
form) og gi det videre til neste infosjef, slik at den da kan ta
utgangspunkt i ting som allerede er laget. Det er ikke meningen å bruke
det samme materialet hvert semester, utformingen kan jo være helt
anderledes, men det er jo ingen vits i å finne opp hjulet på nytt.
Man kan jo også få noen gode ideer på denne måten.

Ved semesterstart i høstsemesteret kan det være lurt å aliere seg med matnat-faddere. Mange av disse er allerede RFere, og bør ikke være noe problem å få kontakt med. Be fadderne opplyse nye studenter om RF, og gi dem gjerne flyers.
