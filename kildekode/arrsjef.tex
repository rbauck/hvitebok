% vim:fileencoding=utf-8
% -*- encoding: utf-8 -*-
\section{Arrsjef}

Realistforeningens (RF) arrsjef har ansvaret for forberedelse og avvikling
av RF's arrangementer.

Arrsjefen leder arrstyret, som i tillegg består av
regiformann, vaktgruppesjef, blæstsjef og arrmesterene. Arrsjefen
skal koordinere denne gruppens arbeid på en slik måte at alle vet hvilke
arrangementer som er forestående, og hva det er forventet at de skal gjøre
i sammenheng med disse.

Arrstyret bestemmer hvilke arrangemter RF skal avholde, og er ansvarlige
for avvikling av disse, f.eks.
kjellerkroer, temafester, konserter o.l., Høstpølse, Psildefrokost og
julebord. I tillegg er arrstyret ansvarlige for Danmarkstur og hyttetur.
Internfester er, med mindre arrsjef i samarbeid med hovedstyret har blitt
enige om det, ikke arrstyrets ansvar.

Under planleggingen av et arrangement skal arrsjefen holde kontakten
mellom de forskjellige grupperingene i arrstyret. Siden arrstyret er
stort og driver med svært forskjellige ting kan dette ofte vise seg
vanskelig, og arrsjefen må derfor være fleksibel i sitt arbeide.

Arrsjefen har ansvaret for å holde kontroll på RF's medlemskapssalg. Dette
innebærer å sørge for at det blir solgt medlemskap på ett arrangement, at
medlemslisten til enhver tid er ajourført, og å sørge for at det finnes
medlemskort.

På ett arrangement er det arrsjefens ansvar å sørge for at det personalet
som kreves for å drive arr. er tilstede. Dette innebærer at det skal være
vakter tilstede (hvis nødvendig), personer til å håndtere lys og lyd (hvis
nødvendig, mannskap til å betjene baren (hvis nødvendig), samt en
hovedansvarlig. Arrsjefens jobb må her anses som gjort hvis de ansvarlige
for disse undergruppene har fått tilstrekkelig instruks i god tid.
Arrsjefen skal ikke lastes for andre grupperingers manglende evne til å
gjøre sine ting.

Før et arrangement skal det også blæstes. Det er arrsjefens jobb å
informere blæstsjefen om hvilke arr som er nært forestående. Blæstsjefen
skal da ha tilstrekkelig informasjon om arr til at han/hun kan blæste arr
skikkelig.

Arrsjefen skal velge ut arrmestere til sin arrgruppe. Disse innstiller han
så for hovedstyret som så stemmer.


\section{Arrmestere}

Arrmestere blir utpekt av arrsjef og innstilt for hovedstyret.
Arrmesterene sitter i arrstyret.

Arrmestere skal assistere arrsjefens i hans oppgaver.

De skal assistere i
planleggingen og forberedelsene til et arrangement, samt hjelpe til under
arrangementer. Arrmesteren skal også jobbe som hovedansvarlig (se egen
instruks) på RF's arrangementer. Arrmestere må derfor ha nødvendig
kunnskap om brannsikkerhet.
