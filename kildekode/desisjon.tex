% vim:fileencoding=utf-8
% -*- encoding: utf-8 -*-
% vim:fileencoding=latin1
% -*- encoding: iso-8859-1 -*-
Fra foreningens lover, §14:
\begin{description}
\item[d] Desisjonsutvalget har den endelige avgjørelse i tvilsspørsmål om
tolkning av lovene. Utvalget kan også fatte vedtak og gi regler i situasjoner
hvor lovene måtte vise seg å være utilstrekkelige. Ethvert medlem av
Realistforeningen har rett til å innanke for Desisjonsutvalget vedtak hvor det
kan være tvil om lovligheten.
\item[e] Ved mistanke om misligheter kan Desisjonsutvalget suspendere
medlemmer av styrer, utvalg og komiteer. Suspensjonen kan omfatte et organ i
sin helhet, selv om det ikke foreligger konkret mistanke mot hvert enkelt
medlem. I tilfelle suspensjon er foretatt skal Desisjonsutvalget straks sørge
for at det blir innkalt til ekstraordinær generalforsamling der
mistillitsforslag behandles og nyvalg eventuelt avholdes.
\item[f] Desisjonsutvalget har ansvar for at Realistforeningens arkiv til 
enhver tid er i orden.
\item[g] Desisjonsutvalget har møte- og talerett i alle foreningens organer.
\end{description}

\section{Lovtolking}
RF er, som kjent, et monarki, og vi praktiserer liten grad av sedvanerett. Derfor har vi
heller ingen fast tolkning av lovene utover hva noen klarer å huske. Tolking av lovene
må derfor skje hver gang noen stiller spørsmål ved noe. Desisjonsutvalget har ved enkelte
anledninger funnet meget romslige tolkninger, men de har likevel passet med lovteksten.

Når det er nødvendig, bør lovene tolkes strengt. Desisjonsutvalgets oppgave er dog å passe på
at foreningen fungerer godt, ikke stikke kjepper i hjulene for styret. Man kan for eksempel
tenke seg at økonomiske rutiner er viktigere enn andre ting.

\subsection{Lovendringer}
Desisjonsutvalget skal uttale seg om lovendringsforslag som er kommet opp på Generalforsamling.
Det er her viktig at Desisjonsutvalget gjør et ordentlig forarbeide, for Desisjonsutvalgets
anbefalinger blir nesten alltid tatt til følge.

Lovendringsforslag skal komme inn senest en uke før Generalforsamling. Etter dette har Desisjonsutvalget
adgang til å komme med forslag inntil 48 timer før. Denne adgangen bør kun brukes til å forbedre
innkomne lovendringsforslag eller til å lage kompromissforslag når det er kommet inn motstridende
forslag. Desisjonsutvalgets medlemmer bør ellers overholde den vanlige fristen for lovendringer.

\section{Suspensjon}
Denne adgangen er gitt Desisjonsutvalget fordi der finnes en kuppekultur blant 
studentforeningene. Desisjonsutvalget kan, hvis for eksempel SVFF skulle kuppe en Generalforsamling,
kaste styret og kalle inn til ny Generalforsamling. Ofte vil det heller være mulig å tolke
medlemsparagrafen på Generalforsamling. Det kan også tenkes andre tilfeller hvor denne adgangen
kan benyttes. Et eksempel er underslag.

\section{Arkivet}
Arkivet ligger i Jomfruburet. Dette er et rom plassert i 5.~etage i fysikkbygningen. For å komme
dit, trenger man en nøkkel. Denne nøkkelen passer til store fysiske lesesal, den østlige døren i 
nordveggen der og til selve Jomfruburet, som er en etage opp trappen fra denne dør.

\section{Møterett}
DU kan stille på alle møter. Av den grunn er det viktig at alle innkallelser kommer i tide. Desisjon
møter av og til når det kan være tvil om lovlighet eller lovspørsmål skal tas opp.

\section{E-post}
Listen desisjon@rf.uio.no skal stå som metamedlem på alle styrelister: as@rf.uio.no, hs@rf.uio.no, ks@rf.uio.no,
blaestgruppa@rf.uio.no, fagstyret@rf.uio.no, gildet@rf.uio.no, okonomi@rf.uio.no og revisjon@rf.uio.no. Desisjonsutvalgets
medlemmer bør unngå å blande seg opp i daglig drift, men det er greit å forklare hvordan ting var før og andre
faktaopplysninger. Disse opplysningene bør være av nøytral karakter.

